\documentclass{article}
% Some basic packages
\usepackage[utf8]{inputenc}
\usepackage[margin=1.2in]{geometry}
\usepackage{textcomp}
\usepackage{url}
\usepackage{graphicx}
\usepackage{float}
\usepackage{enumitem}
\usepackage{standalone}
\usepackage{tcolorbox}
\usepackage{wrapfig}

%color settings
\usepackage{xcolor}
\definecolor{gruvbgdark}{HTML}{1d2021}
\definecolor{gruvtextdark}{HTML}{ebdbb2}
\definecolor{gruvbglight}{HTML}{f9f5d7}
\definecolor{gruvtextlight}{HTML}{3c3836}
% \pagecolor{gruvbgdark}
% \color{gruvtextdark}

% Hide page number when page is empty
\usepackage{emptypage}
\usepackage{subcaption}
\usepackage{multicol}
\usepackage{xcolor}

% Math stuff
\usepackage{amsmath, amsfonts, mathtools, amsthm, amssymb}
% Fancy script capitals
\usepackage{mathrsfs}
\usepackage{cancel}

% Bold math
\usepackage{bm}

% Some shortcuts
\newcommand\N{\ensuremath{\mathbb{N}}}
\newcommand\R{\ensuremath{\mathbb{R}}}
\newcommand\Z{\ensuremath{\mathbb{Z}}}
\renewcommand\O{\ensuremath{\emptyset}}
\newcommand\Q{\ensuremath{\mathbb{Q}}}
\newcommand\C{\ensuremath{\mathbb{C}}}

%Make implies and impliedby shorter
\let\implies\Rightarrow
\let\impliedby\Leftarrow
\let\iff\Leftrightarrow
\let\epsilon\varepsilon

% Add \contra symbol to denote contradiction
% \usepackage{stmaryrd} % for \lightning
% \newcommand\contra{\scalebox{1.5}{$\lightning$}}

% \let\phi\varphi

% Command for short corrections
% Usage: 1+1=\correct{3}{2}

\definecolor{correct}{HTML}{009900}
\newcommand\correct[2]{\ensuremath{\:}{\color{red}{#1}}\ensuremath{\to }{\color{correct}{#2}}\ensuremath{\:}}
\newcommand\green[1]{{\color{correct}{#1}}}

% horizontal rule
% \newcommand\hr{
%     \noindent\rule[0.5ex]{\linewidth}{0.5pt}
% }

% hide parts
\newcommand\hide[1]{}

% Environments
\makeatother
% For box around Definition, Theorem, \ldots
\usepackage{mdframed}
\mdfsetup{skipabove=1em,skipbelow=1em}
\theoremstyle{definition}

\newmdtheoremenv[nobreak=true]{definition}{Definition}
\newtheorem*{eg}{Example}
\newtheorem*{notation}{Notation}
\newtheorem*{previouslyseen}{As previously seen}
\newtheorem*{remark}{Remark}
\newtheorem*{note}{Note}
\newtheorem*{problem}{Problem}
\newtheorem*{observe}{Observe}
\newtheorem*{property}{Property}
\newtheorem*{intuition}{Intuition}
\newmdtheoremenv[nobreak=true]{prop}{Proposition}
\newmdtheoremenv[nobreak=true]{theorem}{Theorem}
\newmdtheoremenv[nobreak=true]{corollary}{Corollary}

% \newtcbtheorem{uctheorem}{Theorem}{uncheckedstyle}{theo}
% End example and intermezzo environments with a small diamond (just like proof
% environments end with a small square)
\usepackage{etoolbox}
\AtEndEnvironment{vb}{\null\hfill$\diamond$}%
\AtEndEnvironment{intermezzo}{\null\hfill$\diamond$}%
% \AtEndEnvironment{opmerking}{\null\hfill$\diamond$}%

% Fix some spacing
% http://tex.stackexchange.com/questions/22119/how-can-i-change-the-spacing-before-theorems-with-amsthm
\makeatletter
\def\thm@space@setup{%
	\thm@preskip=\parskip \thm@postskip=0pt
}


% Exercise 
% Usage:
% \oefening{5}
% \suboefening{1}
% \suboefening{2}
% \suboefening{3}
% gives
% Oefening 5
%   Oefening 5.1
%   Oefening 5.2
%   Oefening 5.3
\newcommand{\oefening}[1]{%
	\def\@oefening{#1}%
	\subsection*{Oefening #1}
}

\newcommand{\suboefening}[1]{%
	\subsubsection*{Oefening \@oefening.#1}
}


% \lecture starts a new lecture (les in dutch)
%
% Usage:
% \lecture{1}{di 12 feb 2019 16:00}{Inleiding}
%
% This adds a section heading with the number / title of the lecture and a
% margin paragraph with the date.

% I use \dateparts here to hide the year (2019). This way, I can easily parse
% the date of each lecture unambiguously while still having a human-friendly
% short format printed to the pdf.

% \usepackage{xifthen}
% \def\testdateparts#1{\dateparts#1\relax}
% \def\dateparts#1 #2 #3 #4 #5\relax{
% 	\marginpar{\small\textsf{\mbox{#1 #2 #3 #5}}}
% }

% \def\@lecture{}%
% \newcommand{\lecture}[3]{
% 	\ifthenelse{\isempty{#3}}{%
% 		\def\@lecture{Lecture #1}%
% 	}{%
% 		\def\@lecture{Lecture #1: #3}%
% 	}%
% 	\subsection*{\@lecture}
% 	% \marginpar{\small\textsf{\mbox{#2}}}
% }



% These are the fancy headers
\usepackage{fancyhdr}
\pagestyle{fancy}

% LE: left even
% RO: right odd
% CE, CO: center even, center odd
% My name for when I print my lecture notes to use for an open book exam.
\fancyhead[LE,RO]{Kristian Sørdal}

\fancyhead[RO,LE]{DAVE3700 - Matte 3000} % Right odd,  Left even
\fancyhead[RE,LO]{\leftmark}          % Right even, Left odd

\fancyfoot[RO,LE]{\thepage}  % Right odd,  Left even
\fancyfoot[RE,LO]{}          % Right even, Left odd
\fancyfoot[C]{\leftmark}     % Center

\makeatother

% Todonotes and inline notes in fancy boxes
\usepackage{todonotes}
\usepackage{tcolorbox}

% Make boxes breakable
\tcbuselibrary{breakable}

% Figure support as explained in my blog post.
\usepackage{import}
\usepackage{xifthen}
\usepackage{pdfpages}
\usepackage{transparent}
\newcommand{\incfig}[1]{%
	\def\svgwidth{\columnwidth}
	\import{./figures/}{#1.pdf_tex}
}

% Fix some stuff
% %http://tex.stackexchange.com/questions/76273/multiple-pdfs-with-page-group-included-in-a-single-page-warning
\pdfsuppresswarningpagegroup=1

\author{Kristian Sørdal}


\begin{document}
\section{Lecture 17}
\subsection{Triple Integrals}

\begin{eg}
	Let \( T \) be the region bounded by the \( xy- \)plane and the paraboloid

	\[ g\left( x,y \right)=4-\left( x^2+y^2 \right) \]

	Compute its volume, that is \( \iiint_{T}dV \). The \( xy \)-plane is \( z=0 \). For \( T \) we have

	\[ 0 \leq z \leq 4-\left( x^2+y^2 \right) \]

	We want to determine the projection of \( T \) in the \( xy- \)plane. We intersect \( z=g\left( x,y \right) \) with \( z=0 \). This means \( g\left( x,y \right)=0 \), and hence

	\[ x^2+y^2=4 \]

	We take all points inside this circle, that is

	\[ D = \left\{ \left( x,y \right):x^2+y^2 \right\} \leq 4 \]

	Its volume is


	\begin{align*}
		\iiint_{T}dV & = \iint_{D}\left[ \int_{0}^{g\left( x,y \right)} dz dA \right] \\
		             & = \iint_{D}\left( 4-x^2-y^2 \right)dA
	\end{align*}

	We compute this in polar coords, the disc \( D \) is described by

	\[ 0 \leq r \leq z, \quad 0 \leq \theta \leq 2\pi \]

	Using \( r^2=x^2+y^2 \), we get

	\begin{align*}
		\iint_{D} \left( 4-x^2-y^2 \right)dA & = \int_{r=0}^{2} \int_{\theta=0}^{2\pi} \left( 4-r^2 \right)r d\theta dr \\
		                                     & = 2\pi \int_{r=0}^{2} \left( 4r-r^3 \right) dr                           \\
		                                     & = 2\pi \left[ 4 \frac{r^2}{2}-\frac{r^{4}}{4} \right]_{0}^{2}            \\
		                                     & = 8\pi
	\end{align*}

	These are analogous formlae for the case

	\begin{align*}
		 & u_{1} \left( y,z \right) \leq x \leq u_{2} \left( y,z \right) \\
		 & u_{1} \left( x,z \right) \leq x \leq u_{2} \left(  x,z\right)
	\end{align*}
\end{eg}

\subsection{Change of Variables}

The general coors \( \left( u,v,w \right) \) are defined by

\[ x = x\left( u,v,w \right), \quad y= y\left( u,v,w \right), \quad z = z\left( u,v,w \right) \]

\begin{definition}
	The Jacobian determinant is

	\[ J\left( u,v,w \right) = \left| \begin{bmatrix}
			x_{u} & x_{v} & x_{w} \\
			y_{u} & y_{v} & y_{w} \\
			z_{u} & z_{u} & z_{w}
		\end{bmatrix} \right|  \]

	Here we have that \( x_{u} = \frac{\partial x}{\partial u} \), and so on.
\end{definition}

\begin{theorem}
	Change of Variables
	\medskip

	Let \( T \) be a region described in \( \left( x,y,z \right) \) coords.


	\begin{figure}[H]
		\centering
		\incfig{tregion}
	\end{figure}

	Let \( T' \) be the corresponding region in \( \left( u,v,w \right) \).


	\begin{figure}[H]
		\centering
		\incfig{uvwtregion}
	\end{figure}

	Then we have

	\[ \int_{T}fdV = \int_{T'}f\left( x\left( u,v,w \right),y\left( u,v,w \right),z\left( u,v,w \right) \right) \cdot \left| J\left( u,v,w \right) \right|dudvdw \]

	We can express this as

	\[ \int_{T}fdV = \int_{T'}fdV' \]

	Where \( dv = dxdydz \) and \( dV' = \left| J\left( u,v,w \right) \right| dudvdw\)
\end{theorem}

\subsection{Cylindrical Coordinates}
\begin{definition}
	The cylindrical coordinates \( \left( r,\theta,x \right) \) are

	\[ x=r\cos\theta, \quad y=r\sin\theta, z=z \]

	Their range are respectively

	\[ 0 \leq r \leq \infty, \quad 0 \leq \theta \leq 2\pi, \quad -\infty \leq z \leq \infty \]


\end{definition}

The geometrical interpretation is shown in the figure below.

\begin{figure}[H]
	\centering
	\incfig{cylindrical}
\end{figure}

\( z \) is the same as in the cartesian case, \( \left( r,\theta \right) \) are the same as in the polar case. How does this relate to cylinders?

\begin{eg}
	We consider a solid cylinder with radius \( t \) and height \( h \).


	\begin{figure}[H]
		\centering
		\incfig{cylinder}
	\end{figure}

	This cylinder is described by

	\[ T = \left\{ \left( x,y,z \right): x^2+y^2 \leq t^2, \quad 0 \leq z \leq h \right\} \]

	Passing to cylindrical coords, the disc \( x^2+y^2 \leq t^2 \) becomes

	\[ 0 \leq r \leq t^2, \quad 0 \leq \theta \leq 2\pi \]

	Then we get the region

	\[ T' = \left\{ \left( r, \theta,z \right): 0 \leq r \leq t,\quad 0 \leq \theta \leq t, \quad 0 \leq z \leq h \right\} \]

\end{eg}

\begin{note}
	In cylindrical coords, the cylinder \( T \) becomes the box \( T' \)


	\begin{figure}[H]
		\centering
		\incfig{boxt}
	\end{figure}
\end{note}

\begin{prop}
	For the cylindrical coords, we have

	\[ dV = rd\theta drdz \]
\end{prop}

\begin{proof}
	The Jacobian is

	\[ J\left( r,\theta,z \right)=\left| \begin{bmatrix}
			\cos\theta & -r\sin\theta & 0 \\
			\sin\theta & r\cos\theta  & 0 \\
			0          & 0            & 1
		\end{bmatrix} \right| \]

	The determinant is

	\[ J\left( r,\theta,z \right)-\left| \begin{bmatrix}
			\cos\theta & -r\sin\theta \\
			\sin\theta & r\cos\theta
		\end{bmatrix} \right|=r \]
\end{proof}

\begin{eg}
	Consider again the cylinder \( T \) as before, then we compute


	\begin{align*}
		\iiint_{T}dV & = \int_{r=0}^{t} \int_{\theta=0}^{2\pi} \int_{z=0}^{h} r dz d\theta dr \\
		             & = 2\pi h \int_{r=0}^{t} rdr                                            \\
		             & = 2\pi h \cdot  \frac{1}{2} t^2                                        \\
		             & = \pi t^2 h
	\end{align*}

	This returns the familiar formula.
\end{eg}

\subsection{Spherical Coordinates}
\begin{definition}
	The spherical coords \( \left( \rho, \phi, \theta \right) \) are

	\[ x=\rho\sin\phi\cos\theta,\quad y=\rho\sin\phi\sin\theta,\quad z=\rho\cos\phi \]

	Their range is

	\[ 0 \leq \rho \leq \infty,\quad 0 \leq \phi \leq \pi, \quad 0 \leq \theta \leq 2\pi \]

	\begin{note}
		Only \( \phi \) goes to \( \pi \).
	\end{note}
\end{definition}


\begin{observe}
	We have the following relations

	\[ \rho = \sqrt{x^2+y^2+z^2}, \tan\theta=\frac{y}{x}  \]
\end{observe}

The geometrical meaning is as follows


\begin{figure}[H]
	\centering
	\incfig{sphericalgeo}
\end{figure}
\end{document}
