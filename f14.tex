\documentclass{article}
% Some basic packages
\usepackage[utf8]{inputenc}
\usepackage[margin=1.2in]{geometry}
\usepackage{textcomp}
\usepackage{url}
\usepackage{graphicx}
\usepackage{float}
\usepackage{enumitem}
\usepackage{standalone}
\usepackage{tcolorbox}
\usepackage{wrapfig}

%color settings
\usepackage{xcolor}
\definecolor{gruvbgdark}{HTML}{1d2021}
\definecolor{gruvtextdark}{HTML}{ebdbb2}
\definecolor{gruvbglight}{HTML}{f9f5d7}
\definecolor{gruvtextlight}{HTML}{3c3836}
% \pagecolor{gruvbgdark}
% \color{gruvtextdark}

% Hide page number when page is empty
\usepackage{emptypage}
\usepackage{subcaption}
\usepackage{multicol}
\usepackage{xcolor}

% Math stuff
\usepackage{amsmath, amsfonts, mathtools, amsthm, amssymb}
% Fancy script capitals
\usepackage{mathrsfs}
\usepackage{cancel}

% Bold math
\usepackage{bm}

% Some shortcuts
\newcommand\N{\ensuremath{\mathbb{N}}}
\newcommand\R{\ensuremath{\mathbb{R}}}
\newcommand\Z{\ensuremath{\mathbb{Z}}}
\renewcommand\O{\ensuremath{\emptyset}}
\newcommand\Q{\ensuremath{\mathbb{Q}}}
\newcommand\C{\ensuremath{\mathbb{C}}}

%Make implies and impliedby shorter
\let\implies\Rightarrow
\let\impliedby\Leftarrow
\let\iff\Leftrightarrow
\let\epsilon\varepsilon

% Add \contra symbol to denote contradiction
% \usepackage{stmaryrd} % for \lightning
% \newcommand\contra{\scalebox{1.5}{$\lightning$}}

% \let\phi\varphi

% Command for short corrections
% Usage: 1+1=\correct{3}{2}

\definecolor{correct}{HTML}{009900}
\newcommand\correct[2]{\ensuremath{\:}{\color{red}{#1}}\ensuremath{\to }{\color{correct}{#2}}\ensuremath{\:}}
\newcommand\green[1]{{\color{correct}{#1}}}

% horizontal rule
% \newcommand\hr{
%     \noindent\rule[0.5ex]{\linewidth}{0.5pt}
% }

% hide parts
\newcommand\hide[1]{}

% Environments
\makeatother
% For box around Definition, Theorem, \ldots
\usepackage{mdframed}
\mdfsetup{skipabove=1em,skipbelow=1em}
\theoremstyle{definition}

\newmdtheoremenv[nobreak=true]{definition}{Definition}
\newtheorem*{eg}{Example}
\newtheorem*{notation}{Notation}
\newtheorem*{previouslyseen}{As previously seen}
\newtheorem*{remark}{Remark}
\newtheorem*{note}{Note}
\newtheorem*{problem}{Problem}
\newtheorem*{observe}{Observe}
\newtheorem*{property}{Property}
\newtheorem*{intuition}{Intuition}
\newmdtheoremenv[nobreak=true]{prop}{Proposition}
\newmdtheoremenv[nobreak=true]{theorem}{Theorem}
\newmdtheoremenv[nobreak=true]{corollary}{Corollary}

% \newtcbtheorem{uctheorem}{Theorem}{uncheckedstyle}{theo}
% End example and intermezzo environments with a small diamond (just like proof
% environments end with a small square)
\usepackage{etoolbox}
\AtEndEnvironment{vb}{\null\hfill$\diamond$}%
\AtEndEnvironment{intermezzo}{\null\hfill$\diamond$}%
% \AtEndEnvironment{opmerking}{\null\hfill$\diamond$}%

% Fix some spacing
% http://tex.stackexchange.com/questions/22119/how-can-i-change-the-spacing-before-theorems-with-amsthm
\makeatletter
\def\thm@space@setup{%
	\thm@preskip=\parskip \thm@postskip=0pt
}


% Exercise 
% Usage:
% \oefening{5}
% \suboefening{1}
% \suboefening{2}
% \suboefening{3}
% gives
% Oefening 5
%   Oefening 5.1
%   Oefening 5.2
%   Oefening 5.3
\newcommand{\oefening}[1]{%
	\def\@oefening{#1}%
	\subsection*{Oefening #1}
}

\newcommand{\suboefening}[1]{%
	\subsubsection*{Oefening \@oefening.#1}
}


% \lecture starts a new lecture (les in dutch)
%
% Usage:
% \lecture{1}{di 12 feb 2019 16:00}{Inleiding}
%
% This adds a section heading with the number / title of the lecture and a
% margin paragraph with the date.

% I use \dateparts here to hide the year (2019). This way, I can easily parse
% the date of each lecture unambiguously while still having a human-friendly
% short format printed to the pdf.

% \usepackage{xifthen}
% \def\testdateparts#1{\dateparts#1\relax}
% \def\dateparts#1 #2 #3 #4 #5\relax{
% 	\marginpar{\small\textsf{\mbox{#1 #2 #3 #5}}}
% }

% \def\@lecture{}%
% \newcommand{\lecture}[3]{
% 	\ifthenelse{\isempty{#3}}{%
% 		\def\@lecture{Lecture #1}%
% 	}{%
% 		\def\@lecture{Lecture #1: #3}%
% 	}%
% 	\subsection*{\@lecture}
% 	% \marginpar{\small\textsf{\mbox{#2}}}
% }



% These are the fancy headers
\usepackage{fancyhdr}
\pagestyle{fancy}

% LE: left even
% RO: right odd
% CE, CO: center even, center odd
% My name for when I print my lecture notes to use for an open book exam.
\fancyhead[LE,RO]{Kristian Sørdal}

\fancyhead[RO,LE]{DAVE3700 - Matte 3000} % Right odd,  Left even
\fancyhead[RE,LO]{\leftmark}          % Right even, Left odd

\fancyfoot[RO,LE]{\thepage}  % Right odd,  Left even
\fancyfoot[RE,LO]{}          % Right even, Left odd
\fancyfoot[C]{\leftmark}     % Center

\makeatother

% Todonotes and inline notes in fancy boxes
\usepackage{todonotes}
\usepackage{tcolorbox}

% Make boxes breakable
\tcbuselibrary{breakable}

% Figure support as explained in my blog post.
\usepackage{import}
\usepackage{xifthen}
\usepackage{pdfpages}
\usepackage{transparent}
\newcommand{\incfig}[1]{%
	\def\svgwidth{\columnwidth}
	\import{./figures/}{#1.pdf_tex}
}

% Fix some stuff
% %http://tex.stackexchange.com/questions/76273/multiple-pdfs-with-page-group-included-in-a-single-page-warning
\pdfsuppresswarningpagegroup=1

\author{Kristian Sørdal}

\begin{document}
\section{Lecture 14}
\subsection{Conservative Feilds (cont.)}

We will now explore other criteria for conservative fields.

\begin{prop}
	Suppose \( \vec{F} \) is conservative, then \( \nabla \times  \vec{F} = 0 \)
\end{prop}

\begin{proof}
	Since \( \vec{F} = \nabla f \), we have

	\[ \nabla \times \vec{F} = \nabla \times \nabla f = 0 \]

	This is by an identity previously discussed.
\end{proof}

The converse for this is also true.

\begin{note}
	It is easy to check if \( \vec{F} \) is conservative by computing \( \nabla \times  \vec{F} \).
\end{note}

\begin{definition}
	A curve is closed if its endpoints coincide.


	\begin{figure}[H]
		\centering
		\incfig[0.5]{endpoints}
	\end{figure}
\end{definition}

\begin{notation}
	The line integral of \( F \) along a closed curve is called the circulation. It is written as

	\[ \oint_{C}\vec{F}\cdot d \vec{r} \]
\end{notation}

\begin{prop}
	If \( \vec{F} \) is conservative, then

	\[ \oint_{C} \vec{F}\cdot  d \vec{r} = 0 \],

	for any closed curve \( C \).
\end{prop}

\begin{proof}
	By the gradient theorem, we have

	\[ \oint_{C} \vec{F}\cdot d \vec{r} = f\left( \vec{p} \right)-f\left( \vec{p} \right)=0 \]

	Since the endpoints coincide
\end{proof}

In summary, we have the following equivalent conditions

\begin{itemize}
	\item \( \vec{F} \) is conservative
	\item \( \vec{F}=\nabla f \)
	\item \( \nabla \times  \vec{F}=0 \)
	\item \( oint_{C}\vec{F}\cdot d \vec{r} = 0 \), for any closed curve \( C \).
\end{itemize}

\subsection{Double Integrals}

In two dimensions, we have the following method for computing integrals


\begin{figure}[H]
	\centering
	\incfig{doubleintegrals}
\end{figure}

We approximate a region \( R \) by rectangles \( R_{i} \), with areas \( \Delta A_{i} \)

Consider a function \( f\left( x,y \right) \), pick a sample point \( \left( x_{i}, y_{i} \right) \) in each rectangle \( R_{i} \). Then we consider the sum

\[ \sum_{i}^{} f\left( x_{i}^{\star},y_{i}^{\star} \right)\Delta A_{i} \]

The limit \( \Delta A_{i} \), when it exists, gives the double integral.

\begin{definition}
	The double integral of \( f\left( x,y \right) \) over the region \( R \) is

	\[ \iint_{R}fdA = \lim_{\Delta A_{i}\rightarrow 0} \sum_{n=i}^{} f\left( x_{i}^{\star},y_{i}^{\star} \right)\Delta A_{i}  \]
\end{definition}

When \( f=1 \), this gives the area of \( R \), or the size of the region \( R \). When \( f > 0 \), the integral is also the volume under \( f \).

\subsection{Some Properties}

We still need concrete forumlas to compute \( \iint_{R}fdA \). First, some general properties.

\begin{prop}
	Linearity.
	\medskip

	Let \( a \) and \( b \) be two constants, then

	\[ \int_{R}\left( af+bg \right)dA = a \int_{R}fdA+b \int_{R}gdA \]
\end{prop}

\begin{proof}
	This follows the linearity of limits.
\end{proof}

The next property is related to portions of the region of integrations.


\begin{figure}[H]
	\centering
	\incfig{overlappingregions}
\end{figure}

\begin{prop}
	Partitions
	\medskip

	Let \( R \text{ and } S \) be non-overlapping regions. Then we have

	\[ \int_{R\cup B}fdA = \int_{R}fdA + \int_{S}fdA \]
\end{prop}

\begin{idea}
	The total area is the sum of the areas.
\end{idea}
\subsection{Integrations Over Rectangles}

Integrations over a rectangle is the easiest case of a double integral.


\begin{figure}[H]
	\centering
	\incfig{integralrectangle}
\end{figure}

General rectangle:

\[ R = \left( a,b \right)\times \left( c,d \right) \]

\begin{prop}
	Let \( R = \left( a,b \right)\times \left( c,d \right) \), then


	\[ \iint_{R}fdA = \int_{a}^{b} \left( \int_{c}^{d} f\left( x,y \right) dy \right) dx = \int_{c}^{d} \left( \int_{a}^{b} f\left( x,y \right) dx \right) dy \]

	We reduce to the case of two ordinary integrals.

\end{prop}

\begin{eg}
	By simple geometry, the area of a rectangle is \( \left( b-a \right)\cdot \left( d-c \right) \). The double integral gives

	\[ \iint_{R}1dA = \int_{a}^{b} \left( \int_{c}^{d} 1 dy \right) dx = \int_{a}^{b} \left( d-c \right) dx = \left( b-a \right)\cdot \left( d-c \right)\]

\end{eg}

\begin{eg}
	Compute \( \iint_{R}fdA \) with

	\[ f\left( x,y \right)=xy,\quad R = \left[ 0,1 \right]\times \left[ 0,2 \right] \]

	We compute
\end{eg}
\end{document}
