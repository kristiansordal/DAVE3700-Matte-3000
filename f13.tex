\documentclass{article}
% Some basic packages
\usepackage[utf8]{inputenc}
\usepackage[margin=1.2in]{geometry}
\usepackage{textcomp}
\usepackage{url}
\usepackage{graphicx}
\usepackage{float}
\usepackage{enumitem}
\usepackage{standalone}
\usepackage{tcolorbox}
\usepackage{wrapfig}

%color settings
\usepackage{xcolor}
\definecolor{gruvbgdark}{HTML}{1d2021}
\definecolor{gruvtextdark}{HTML}{ebdbb2}
\definecolor{gruvbglight}{HTML}{f9f5d7}
\definecolor{gruvtextlight}{HTML}{3c3836}
% \pagecolor{gruvbgdark}
% \color{gruvtextdark}

% Hide page number when page is empty
\usepackage{emptypage}
\usepackage{subcaption}
\usepackage{multicol}
\usepackage{xcolor}

% Math stuff
\usepackage{amsmath, amsfonts, mathtools, amsthm, amssymb}
% Fancy script capitals
\usepackage{mathrsfs}
\usepackage{cancel}

% Bold math
\usepackage{bm}

% Some shortcuts
\newcommand\N{\ensuremath{\mathbb{N}}}
\newcommand\R{\ensuremath{\mathbb{R}}}
\newcommand\Z{\ensuremath{\mathbb{Z}}}
\renewcommand\O{\ensuremath{\emptyset}}
\newcommand\Q{\ensuremath{\mathbb{Q}}}
\newcommand\C{\ensuremath{\mathbb{C}}}

%Make implies and impliedby shorter
\let\implies\Rightarrow
\let\impliedby\Leftarrow
\let\iff\Leftrightarrow
\let\epsilon\varepsilon

% Add \contra symbol to denote contradiction
% \usepackage{stmaryrd} % for \lightning
% \newcommand\contra{\scalebox{1.5}{$\lightning$}}

% \let\phi\varphi

% Command for short corrections
% Usage: 1+1=\correct{3}{2}

\definecolor{correct}{HTML}{009900}
\newcommand\correct[2]{\ensuremath{\:}{\color{red}{#1}}\ensuremath{\to }{\color{correct}{#2}}\ensuremath{\:}}
\newcommand\green[1]{{\color{correct}{#1}}}

% horizontal rule
% \newcommand\hr{
%     \noindent\rule[0.5ex]{\linewidth}{0.5pt}
% }

% hide parts
\newcommand\hide[1]{}

% Environments
\makeatother
% For box around Definition, Theorem, \ldots
\usepackage{mdframed}
\mdfsetup{skipabove=1em,skipbelow=1em}
\theoremstyle{definition}

\newmdtheoremenv[nobreak=true]{definition}{Definition}
\newtheorem*{eg}{Example}
\newtheorem*{notation}{Notation}
\newtheorem*{previouslyseen}{As previously seen}
\newtheorem*{remark}{Remark}
\newtheorem*{note}{Note}
\newtheorem*{problem}{Problem}
\newtheorem*{observe}{Observe}
\newtheorem*{property}{Property}
\newtheorem*{intuition}{Intuition}
\newmdtheoremenv[nobreak=true]{prop}{Proposition}
\newmdtheoremenv[nobreak=true]{theorem}{Theorem}
\newmdtheoremenv[nobreak=true]{corollary}{Corollary}

% \newtcbtheorem{uctheorem}{Theorem}{uncheckedstyle}{theo}
% End example and intermezzo environments with a small diamond (just like proof
% environments end with a small square)
\usepackage{etoolbox}
\AtEndEnvironment{vb}{\null\hfill$\diamond$}%
\AtEndEnvironment{intermezzo}{\null\hfill$\diamond$}%
% \AtEndEnvironment{opmerking}{\null\hfill$\diamond$}%

% Fix some spacing
% http://tex.stackexchange.com/questions/22119/how-can-i-change-the-spacing-before-theorems-with-amsthm
\makeatletter
\def\thm@space@setup{%
	\thm@preskip=\parskip \thm@postskip=0pt
}


% Exercise 
% Usage:
% \oefening{5}
% \suboefening{1}
% \suboefening{2}
% \suboefening{3}
% gives
% Oefening 5
%   Oefening 5.1
%   Oefening 5.2
%   Oefening 5.3
\newcommand{\oefening}[1]{%
	\def\@oefening{#1}%
	\subsection*{Oefening #1}
}

\newcommand{\suboefening}[1]{%
	\subsubsection*{Oefening \@oefening.#1}
}


% \lecture starts a new lecture (les in dutch)
%
% Usage:
% \lecture{1}{di 12 feb 2019 16:00}{Inleiding}
%
% This adds a section heading with the number / title of the lecture and a
% margin paragraph with the date.

% I use \dateparts here to hide the year (2019). This way, I can easily parse
% the date of each lecture unambiguously while still having a human-friendly
% short format printed to the pdf.

% \usepackage{xifthen}
% \def\testdateparts#1{\dateparts#1\relax}
% \def\dateparts#1 #2 #3 #4 #5\relax{
% 	\marginpar{\small\textsf{\mbox{#1 #2 #3 #5}}}
% }

% \def\@lecture{}%
% \newcommand{\lecture}[3]{
% 	\ifthenelse{\isempty{#3}}{%
% 		\def\@lecture{Lecture #1}%
% 	}{%
% 		\def\@lecture{Lecture #1: #3}%
% 	}%
% 	\subsection*{\@lecture}
% 	% \marginpar{\small\textsf{\mbox{#2}}}
% }



% These are the fancy headers
\usepackage{fancyhdr}
\pagestyle{fancy}

% LE: left even
% RO: right odd
% CE, CO: center even, center odd
% My name for when I print my lecture notes to use for an open book exam.
\fancyhead[LE,RO]{Kristian Sørdal}

\fancyhead[RO,LE]{DAVE3700 - Matte 3000} % Right odd,  Left even
\fancyhead[RE,LO]{\leftmark}          % Right even, Left odd

\fancyfoot[RO,LE]{\thepage}  % Right odd,  Left even
\fancyfoot[RE,LO]{}          % Right even, Left odd
\fancyfoot[C]{\leftmark}     % Center

\makeatother

% Todonotes and inline notes in fancy boxes
\usepackage{todonotes}
\usepackage{tcolorbox}

% Make boxes breakable
\tcbuselibrary{breakable}

% Figure support as explained in my blog post.
\usepackage{import}
\usepackage{xifthen}
\usepackage{pdfpages}
\usepackage{transparent}
\newcommand{\incfig}[1]{%
	\def\svgwidth{\columnwidth}
	\import{./figures/}{#1.pdf_tex}
}

% Fix some stuff
% %http://tex.stackexchange.com/questions/76273/multiple-pdfs-with-page-group-included-in-a-single-page-warning
\pdfsuppresswarningpagegroup=1

\author{Kristian Sørdal}


\begin{document}
\section{Lecture 13}
\subsection{Line Integrals of Vector Fields}
We have seen the following definition earlier

\[ \int_{C}\vec{F}d \vec{r} = \int_{t_{0} }^{t_{1} } \vec{F}\left( \vec{r}\left( t \right)\right) \cdot  \vec{r}'\left( t  \right)dt \]

We can rewrite this to link with scalarfields, lets consider

\[ \int_{C} = \vec{F}\cdot d \vec{r} = \int_{t_{0} }^{t_{1} } \vec{F}\cdot \vec{v} dt = \int_{t_{0} }^{t_{1} } \vec{F}\cdot \frac{\vec{v}}{\left| \vec{v} \right|}  dt \]

We can see this as the line integral of the field \( \vec{F}\cdot \vec{T} \)

\begin{eg}
	Consider \( \vec{F} \) is constant and directed along the curve, that is \( \vec{F}=F\vec{T} \), where \( F - \left| \vec{F} \right| \). Then the formula for elementary work (\( W=FS \)) gives

	\[ W=\int_{C}\vec{F}\cdot \vec{T}ds=\int_{C}\vec{F}\vec{T}\cdot \vec{T}ds=F \int_{C}ds=FS \]
\end{eg}

Now lets consider changes of orientation

\begin{prop}
	We have that

	\[ \int_{C}\vec{F} \cdot d \vec{r} = -\int_{-C}\vec{F} \cdot  d \vec{r} \]

\end{prop}

\begin{eg}
	Consider the segment

	\[ \vec{r}\left( t \right) = \left( t,0 \right), \quad 0 \leq t \leq 1 \]


	\begin{figure}[H]
		\centering
		\incfig[0.5]{f13eg1}
	\end{figure}

	The opposite parametrization is

	\[ \vec{r}_{\text{opp}}\left( t \right)=\left( -t,0 \right),\quad -1 \leq t \leq 0 \]


	\begin{figure}[H]
		\centering
		\incfig[0.5]{f13eg1.2}
	\end{figure}

	We have

	\[ \vec{r}'\left( t \right)=\left( 1,0 \right),\quad \vec{r}_{\text{opp}}\left( t \right)=\left( -1,0 \right) \]

	Consider \( \vec{F}\left( x,y \right)=\left( x,0 \right) \). For \( C \) we have

	\[ \int_{C}\vec{F} \cdot d \vec{r} = \int_{0}^{1} \left( t,0 \right)\cdot \left( 1,0 \right) dt = \int_{0}^{1} t dt=\frac{1}{2}  \]

	For \( -C \), we have

	\[ \int_{-C}\vec{F} \cdot d \vec{r} = \int_{-1}^{0} \left( -t,0 \right)\cdot \left( -1,0 \right) dt=\int_{-1}^{0} t dt=-\frac{1}{2}  \]
\end{eg}

\subsection{Conservative Vector Fields}
In general, \( \int_{C}\vec{F}\cdot d \vec{r} \) depends on the curve \( C \). However, sometimes it only depends on the endpoints.

\begin{definition}
	A vector field \( \vec{F} \) is conservative if \( \int_{C}\vec{F}\cdot d \vec{r} \) depends only on the endpoints of \( C \). That is, if \( C \) and \( C' \) have the same endpoints, then

	\[ \int_{C}\vec{F} \cdot  d \vec{r} = \int_{C'}\vec{F} \cdot  d \vec{r} \]


	\begin{figure}[H]
		\centering
		\incfig[0.6]{f13def1}
	\end{figure}
\end{definition}

\begin{eg}
	Consider \( \vec{F}=\left( 1,1 \right) \) and

	\[ C: x\left( t \right)=t,\quad y\left( t \right)=0,\quad 0 \leq t \leq 1 \]

	\begin{figure}[H]
		\centering
		\incfig[0.5]{f13eg1}
	\end{figure}

	The endpoints are \( \left( 0,0 \right) \) and \( \left( 1,0 \right) \). We compute

	\[ \int_{C}\vec{F} \cdot  d \vec{r} = \int_{0}^{1} dt =1 \]

	Now, lets consider a different curve.

	\[ C': x\left( t \right)=t, \quad y\left( t \right)=t\left( t-1 \right), \quad 0 \leq t \leq 1 \]

	\begin{figure}[H]
		\centering
		\incfig[0.5]{f13eg2}
	\end{figure}

	We have the same endpoints as \( C \). The velocity is \( \vec{r}'\left( t \right)=\left( 1,2t-1 \right) \). Then


	\begin{align*}
		\int_{C'}\vec{F} \cdot  d \vec{r} & = \int_{0}^{1} \left( 1,1 \right) \cdot  \left( 1, 2t-1 \right) dt \\
		                                  & = \int_{0}^{1} 2tdt - 2 \cdot  \frac{1}{2}  = 1
	\end{align*}
	At this stage, we cannot conclude \( \vec{F} \) is consrvative (although it is). Note that

	\[ \vec{F}=\nabla f \quad\wedge\quad f\left( x,y \right)=x+y \]

	We will prove that being a gradient field is the condition we want.
\end{eg}

\begin{theorem}
	\textbf{The Gradient Theorem}.
	\medskip

	Suppose \( \vec{F}= \nabla f \), consider a curve \( C \) starting at \( \vec{p} \), and ending at \( \vec{q} \), then

	\[ \int_{C}\vec{F} \cdot d \vec{r} = f\left( \vec{q} \right)-f\left( \vec{p} \right) \]


	\begin{figure}[H]
		\centering
		\incfig[0.6]{gradienttheorem}
	\end{figure}
\end{theorem}

\begin{proof}
	Pick a parametrization

	\[ \vec{r}\left( t \right),\quad t_{0} \leq t \leq t_{1}  \]

	Note that: \( \vec{r}\left( t_{0}  \right)=\vec{p} \) and \( r\left( t_{1}  \right)=\vec{q} \). Using \( \vec{F}=\nabla f \), we have


	\begin{align*}
		\int_{C}\vec{F} \cdot  d \vec{r} & = \int_{t_{0} }^{t_{1} } \vec{F}\left( \vec{r}\left( t \right) \right)\cdot  \vec{r}' \left( t \right)dt  \\
		                                 & = \int_{t_{0} }^{t_{1} } \nabla f\left( \vec{r}\left( t \right) \right)\cdot  \vec{r}'\left( t \right) dt
	\end{align*}

	From the chain rule, we get

	\[ \frac{d f\left( \vec{r}\left( t \right) \right)}{dt}=\nabla f\left( \vec{r}\left( t \right) \right)\cdot \vec{r}'\left( t \right) \]

	Using this, we obtain

	\[ \int_{C}\vec{F}\cdot d \vec{r} = \int_{t_{0} }^{t_{1} } \frac{d f\left( \vec{r}\left( t \right) \right)}{dt} dt \]

	Using the fundamental theorem of calculus, we get


	\begin{align*}
		\int_{C}\vec{F} \cdot  d \vec{r} & = f\left( \vec{r}\left( t_{1}  \right) \right) - f\left( \vec{r}\left( t_{0}  \right) \right) \\
		                                 & = f\left( \vec{q} \right)-f\left( \vec{p} \right)
	\end{align*}

\end{proof}

If \( \vec{F}=\nabla f \), then \( \vec{F} \) is conservative, since \( \int_{C}\vec{F}\cdot d \vec{r} = f\left( \vec{q}\right) - f\left( \vec{p}  \right) \) only depends on \( \vec{p} \text{ and } \vec{q} \)

\begin{eg}
	Consider the previous example with \( \vec{F}=\left( 1,1 \right) \). We saw \( \vec{F}= \nabla f \) with \( f\left( x,y \right)=x+y \).
	\medskip

	For any curve starting at \( \vec{p}=\left( 0,0 \right) \), and ending at \( \vec{q}=\left( 1,0 \right) \), we have

	\begin{align*}
		\int_{C}\vec{F}\cdot d \vec{r} & = f\left( \vec{q} \right) - f\left( \vec{p} \right) \\
		                               & = f\left( 1,0 \right)-f\left( 0,0 \right)           \\
		                               & = 1-0 = 1
	\end{align*}

	Note that we could take

	\[ \tilde{f}\left( x,y \right)=x+y+c \]

	With \( c \) being a constant. This gives the same result.

	\begin{align*}
		\int_{C}\vec{F} \cdot  d \vec{r} & = \tilde{f}\left( 1,0 \right) - \tilde{f}\left( 0,0 \right) \\
		                                 & = 1 + c - c = 0
	\end{align*}
\end{eg}

\begin{eg}
	Consider two objects \( A \text{ and } B \) of mass \( M \text{ and } m \), as in the picutre


	\begin{figure}[H]
		\centering
		\incfig{objectmass}
	\end{figure}

	The force exerted by \( A \) on \( B \) is

	\[ \vec{F}\left( x,y,z \right)=-G \frac{Mm}{r^2}\hat{r} \]

	Here we have \( r=\sqrt{x^2+y^2+z^2} \) ant \( \hat{r}=\frac{\left( x,y,z \right)}{r} \). It is a unit vector pointing at \( B \), from \( A \). \( \vec{F} \), can be rewritten as

	\[ \vec{F}\left( x,y,z \right)=-GMm \frac{\left( x,y,z \right)\left( x^2+y^2+z^2 \right)^{\frac{3}{2} }}{} \]

	Consider the function

	\[ V = \frac{GMm}{r} = GMm \frac{1}{\left( x^2+y^2+z^2 \right)^{\frac{1}{2} }} \]

	We compute

	\[ \frac{\partial V}{\partial x}-GMm \frac{1}{2} \cdot  \frac{2x}{\left( x^2+y^2+z^2 \right)^{\frac{3}{2} }} \]

	Similarily for \( y \) and \( z \), then

	\[ \nabla V = -GMm \frac{\left( x,y,z \right)}{\left( x^2+y^2+z^2 \right)^{\frac{3}{2} }} \]
\end{eg}


\end{document}
