\documentclass{article}
% Some basic packages
\usepackage[utf8]{inputenc}
\usepackage[margin=1.2in]{geometry}
\usepackage{textcomp}
\usepackage{url}
\usepackage{graphicx}
\usepackage{float}
\usepackage{enumitem}
\usepackage{standalone}
\usepackage{tcolorbox}
\usepackage{wrapfig}

%color settings
\usepackage{xcolor}
\definecolor{gruvbgdark}{HTML}{1d2021}
\definecolor{gruvtextdark}{HTML}{ebdbb2}
\definecolor{gruvbglight}{HTML}{f9f5d7}
\definecolor{gruvtextlight}{HTML}{3c3836}
% \pagecolor{gruvbgdark}
% \color{gruvtextdark}

% Hide page number when page is empty
\usepackage{emptypage}
\usepackage{subcaption}
\usepackage{multicol}
\usepackage{xcolor}

% Math stuff
\usepackage{amsmath, amsfonts, mathtools, amsthm, amssymb}
% Fancy script capitals
\usepackage{mathrsfs}
\usepackage{cancel}

% Bold math
\usepackage{bm}

% Some shortcuts
\newcommand\N{\ensuremath{\mathbb{N}}}
\newcommand\R{\ensuremath{\mathbb{R}}}
\newcommand\Z{\ensuremath{\mathbb{Z}}}
\renewcommand\O{\ensuremath{\emptyset}}
\newcommand\Q{\ensuremath{\mathbb{Q}}}
\newcommand\C{\ensuremath{\mathbb{C}}}

%Make implies and impliedby shorter
\let\implies\Rightarrow
\let\impliedby\Leftarrow
\let\iff\Leftrightarrow
\let\epsilon\varepsilon

% Add \contra symbol to denote contradiction
% \usepackage{stmaryrd} % for \lightning
% \newcommand\contra{\scalebox{1.5}{$\lightning$}}

% \let\phi\varphi

% Command for short corrections
% Usage: 1+1=\correct{3}{2}

\definecolor{correct}{HTML}{009900}
\newcommand\correct[2]{\ensuremath{\:}{\color{red}{#1}}\ensuremath{\to }{\color{correct}{#2}}\ensuremath{\:}}
\newcommand\green[1]{{\color{correct}{#1}}}

% horizontal rule
% \newcommand\hr{
%     \noindent\rule[0.5ex]{\linewidth}{0.5pt}
% }

% hide parts
\newcommand\hide[1]{}

% Environments
\makeatother
% For box around Definition, Theorem, \ldots
\usepackage{mdframed}
\mdfsetup{skipabove=1em,skipbelow=1em}
\theoremstyle{definition}

\newmdtheoremenv[nobreak=true]{definition}{Definition}
\newtheorem*{eg}{Example}
\newtheorem*{notation}{Notation}
\newtheorem*{previouslyseen}{As previously seen}
\newtheorem*{remark}{Remark}
\newtheorem*{note}{Note}
\newtheorem*{problem}{Problem}
\newtheorem*{observe}{Observe}
\newtheorem*{property}{Property}
\newtheorem*{intuition}{Intuition}
\newmdtheoremenv[nobreak=true]{prop}{Proposition}
\newmdtheoremenv[nobreak=true]{theorem}{Theorem}
\newmdtheoremenv[nobreak=true]{corollary}{Corollary}

% \newtcbtheorem{uctheorem}{Theorem}{uncheckedstyle}{theo}
% End example and intermezzo environments with a small diamond (just like proof
% environments end with a small square)
\usepackage{etoolbox}
\AtEndEnvironment{vb}{\null\hfill$\diamond$}%
\AtEndEnvironment{intermezzo}{\null\hfill$\diamond$}%
% \AtEndEnvironment{opmerking}{\null\hfill$\diamond$}%

% Fix some spacing
% http://tex.stackexchange.com/questions/22119/how-can-i-change-the-spacing-before-theorems-with-amsthm
\makeatletter
\def\thm@space@setup{%
	\thm@preskip=\parskip \thm@postskip=0pt
}


% Exercise 
% Usage:
% \oefening{5}
% \suboefening{1}
% \suboefening{2}
% \suboefening{3}
% gives
% Oefening 5
%   Oefening 5.1
%   Oefening 5.2
%   Oefening 5.3
\newcommand{\oefening}[1]{%
	\def\@oefening{#1}%
	\subsection*{Oefening #1}
}

\newcommand{\suboefening}[1]{%
	\subsubsection*{Oefening \@oefening.#1}
}


% \lecture starts a new lecture (les in dutch)
%
% Usage:
% \lecture{1}{di 12 feb 2019 16:00}{Inleiding}
%
% This adds a section heading with the number / title of the lecture and a
% margin paragraph with the date.

% I use \dateparts here to hide the year (2019). This way, I can easily parse
% the date of each lecture unambiguously while still having a human-friendly
% short format printed to the pdf.

% \usepackage{xifthen}
% \def\testdateparts#1{\dateparts#1\relax}
% \def\dateparts#1 #2 #3 #4 #5\relax{
% 	\marginpar{\small\textsf{\mbox{#1 #2 #3 #5}}}
% }

% \def\@lecture{}%
% \newcommand{\lecture}[3]{
% 	\ifthenelse{\isempty{#3}}{%
% 		\def\@lecture{Lecture #1}%
% 	}{%
% 		\def\@lecture{Lecture #1: #3}%
% 	}%
% 	\subsection*{\@lecture}
% 	% \marginpar{\small\textsf{\mbox{#2}}}
% }



% These are the fancy headers
\usepackage{fancyhdr}
\pagestyle{fancy}

% LE: left even
% RO: right odd
% CE, CO: center even, center odd
% My name for when I print my lecture notes to use for an open book exam.
\fancyhead[LE,RO]{Kristian Sørdal}

\fancyhead[RO,LE]{DAVE3700 - Matte 3000} % Right odd,  Left even
\fancyhead[RE,LO]{\leftmark}          % Right even, Left odd

\fancyfoot[RO,LE]{\thepage}  % Right odd,  Left even
\fancyfoot[RE,LO]{}          % Right even, Left odd
\fancyfoot[C]{\leftmark}     % Center

\makeatother

% Todonotes and inline notes in fancy boxes
\usepackage{todonotes}
\usepackage{tcolorbox}

% Make boxes breakable
\tcbuselibrary{breakable}

% Figure support as explained in my blog post.
\usepackage{import}
\usepackage{xifthen}
\usepackage{pdfpages}
\usepackage{transparent}
\newcommand{\incfig}[1]{%
	\def\svgwidth{\columnwidth}
	\import{./figures/}{#1.pdf_tex}
}

% Fix some stuff
% %http://tex.stackexchange.com/questions/76273/multiple-pdfs-with-page-group-included-in-a-single-page-warning
\pdfsuppresswarningpagegroup=1

\author{Kristian Sørdal}


\begin{document}
\section{Lecture 7}

\subsection{Parametrized Curve}

A curve is described as a sset of points in \( \R^2 \text{ and } \R^3 \). For instance a line is described by

\[ f = \left\{ \left( x,y \right) \in \R^2:x=y \right\} \]

\begin{figure}[H]
	\centering
	\def\svgwidth{1\textwidth}
	\import{./figures/}{fig8.pdf_tex}
\end{figure}

This is a static picture. But how do we give a dynamical picture? We'll use parametrized curves.

\begin{definition}
	A parametrization of a curve \( c \) in \( \R^2 \), is given by

	\[ \vec{r}\left( t \right) = \left( x\left( t \right), y\left( t \right) \right), \quad t_{0} \leq t \leq t_{1}  \]

	Such that \( \vec{r}\left( t \right) \in c \) for all time \( t \).
\end{definition}

\begin{itemize}
	\item A parametrization describes motion. Think of \( t \) as the time.
	\item A parametrization is not unique.
	\item Various natural assumptions, such as continuity and differentiability.
\end{itemize}
\medskip

\begin{eg}
	Consider the funtion \( \vec{r}\left( t \right) \) with \( 0 \leq t \leq 1 \).

	\begin{figure}[H]
		\centering
		\def\svgwidth{0.55\textwidth}
		\import{./figures/}{fig9.pdf_tex}
	\end{figure}

	\[ \vec{r}\left( 0 \right) = \left( 0,0 \right),\quad \vec{r}\left( 1 \right)=\left( 1,1 \right) \]

	We have the portion of the line where \( y=x \). Notce that here, \( x\left( t \right)=t \), \( y\left( t \right)=t \) and \( y\left( t \right)=x\left( t \right)=t \) for all \( t \).

	Lets consider a different function,

	\[ \vec{r}\left( t \right)=\left( 2t,2t \right), \quad 0 \leq t \leq \frac{1}{2}  \]

	When we parametrize, we get

	\[ \vec{r}\left( 0 \right) = \left( 0,0 \right), \quad \vec{r}\left( \frac{1}{2}  \right) = \left( 1,1 \right) \]

	We are moving along the curve at twice the speed.
\end{eg}
\medskip

\begin{eg}
	Given \( f(x) \), we consider

	\[ \vec{r}\left( t \right) = \left( t, f\left( t \right) \right),\quad t_{0} \leq t \leq t_{1}  \]

	This describes a portion of the graph \( f \), with

	\[ \text{Start: } \left( t_{0} ,f\left( t_{0}  \right) \right),\quad \text{End: } \left( t_{1}, f\left( t_{1}  \right) \right) \]

	For instance, consider the line \( y = mx + c \), we have

	\begin{figure}[H]
		\centering
		\def\svgwidth{1\textwidth}
		\import{./figures/}{fig10.pdf_tex}
	\end{figure}

	We have that

	\[ \vec{r}\left( t \right) = \left( t, mt + c \right), \quad t_{0} \leq t \leq t_{1}  \]

\end{eg}
\medskip

\begin{eg}
	We want to describe a line with

	\[ \text{Start: } A = \left( x_{0} ,y_{0}  \right), \quad \text{End: } B = \left( x_{1} , y_{1}  \right) \]

	Then we take the parametrization

	\[ \vec{r}\left( t \right) = \left( 1-t \right)A + tB, \quad 0 \leq t \leq 1 \]

	More explicitly, we have

	\[ \vec{r}\left( t \right) = \left( \left( 1-t \right)x_{0} + tx, \left( 1-t \right)y_{0} + ty \right) \]

	Note that \( \vec{r}\left( 0 \right) = A \text{ and } \vec{r}\left( 1 \right)=B\).
\end{eg}
\medskip

\begin{eg}
	Consider \( \vec{r}\left( t \right) = \left( \cos t, \sin t \right), 0 \leq t \leq 2\pi \). What curve does this describe?

	\begin{figure}[H]
		\centering
		\def\svgwidth{1\textwidth}
		\import{./figures/}{fig11.pdf_tex}
	\end{figure}

	It describes a circle.

	\[ x\left( t \right)^2 + y\left( t \right)^2 = \left( \cos\left( t \right)^2+\sin\left( t \right)^2 \right)=1 \]

	We start at \( \left( 1,0 \right)  \) and move counter-clockwise.
\end{eg}
\medskip

\begin{eg}
	Here is an example from physics. Consider

	\[ x\left( t \right)=v_{x}t, \quad y\left( t \right) = v_{y}t-\frac{1}{2} gt^2, \quad 0 \leq t \leq \frac{2v_{y}}{g}  \]

	This describes the motion of an object with initial velocity \( \vec{v} = \left( v_{x}, v_{y} \right) \), under gravity. We write \( t_{0} = 0 \) and \( t_{1} = \frac{2v_{y}}{g} \). Note that

	\[ \vec{r}\left( t_{0}  \right)= \left( 0,0 \right), \quad \vec{r}\left( t_{1}  \right) = \left( \frac{2v_{x}v_{y}}{g}, 0 \right) \]

	The object falls back to the ground at time \( t_{1}  \).

	\begin{figure}[H]
		\centering
		\def\svgwidth{1\textwidth}
		\import{./figures/}{fig12.pdf_tex}
	\end{figure}

	Well known fact: This motion is parabolic, we will rederive this.
	\medskip

	From \( x\left( t \right) = v_{x}t \), we get \( t = \frac{x\left( t \right)}{v_{x}} \). Then

	\[ y\left( t \right) = v_{y}t-\frac{1}{2} gt^2 \Rightarrow \frac{v_{x}}{v_{y}}x\left( t \right) - \frac{1}{2} \frac{g}{v_{x}^2}   x\left( t \right)^2\]

	This is the expression of a parabola

	\[ y = ax^2 + bx +c,\quad a \neq 0 \]

	It can also be written as

	\[ y\left( t \right) = -\frac{1}{2} \frac{g}{v_{x}^2} \left( x\left( t \right) - \frac{v_{y}}{g} \right)^2 + \frac{1}{2} \frac{v_{y}}{g} \frac{v_{y}}{v_{x}} \]
\end{eg}
\medskip

\subsection{Kinematics}
Kinematics describes position, velocity and acceleration of an object.

\begin{definition}
	The position vector is \( \vec{r}\left( t \right) \). The velocity vector is \( \vec{v}\left( t \right)\frac{d \vec{r}}{dt} \). The acceleration vector is \( \vec{a}\left( t \right) = \frac{d^2\vec{r} }{dt^2} \).
\end{definition}

If we write \( \vec{r}\left( t \right)=\left( x\left( t \right),y\left( t \right) \right) \), then

\[ \vec{v}\left( t \right) = \left( \frac{d x}{dt}, \frac{d y}{dt} \right) = \left( x'\left( t \right), y\left( t \right) \right) \]

Similarily

\[ \vec{a}\left( t \right) = \left( \frac{d ^2 x}{dt^2}, \frac{d ^2 y}{dt^2} \right) = \left( x''\left( t \right), y''\left( t \right) \right) \]

\begin{eg}
	Consider again the gravity example. Here we have

	\[ \vec{r}\left( t \right) = \left( v_{x}t, v_{y}t-\frac{1}{2} gt^2 \right)\]

	The velocity is

	\[ \vec{v}\left( t \right)=\left( v_{x}, v_{y}-gt \right) \]

	Note that \( v\left( 0 \right)=\left( v_{x}, v_{y} \right) \) is the initial velocity of the object. For acceleration, we get

	\[ \vec{a}\left( t \right) = \frac{d \vec{v}}{dt} = \left( 0,-g \right) \]
\end{eg}



\end{document}
