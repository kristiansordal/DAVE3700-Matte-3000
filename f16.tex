\documentclass{article}
% Some basic packages
\usepackage[utf8]{inputenc}
\usepackage[margin=1.2in]{geometry}
\usepackage{textcomp}
\usepackage{url}
\usepackage{graphicx}
\usepackage{float}
\usepackage{enumitem}
\usepackage{standalone}
\usepackage{tcolorbox}
\usepackage{wrapfig}

%color settings
\usepackage{xcolor}
\definecolor{gruvbgdark}{HTML}{1d2021}
\definecolor{gruvtextdark}{HTML}{ebdbb2}
\definecolor{gruvbglight}{HTML}{f9f5d7}
\definecolor{gruvtextlight}{HTML}{3c3836}
% \pagecolor{gruvbgdark}
% \color{gruvtextdark}

% Hide page number when page is empty
\usepackage{emptypage}
\usepackage{subcaption}
\usepackage{multicol}
\usepackage{xcolor}

% Math stuff
\usepackage{amsmath, amsfonts, mathtools, amsthm, amssymb}
% Fancy script capitals
\usepackage{mathrsfs}
\usepackage{cancel}

% Bold math
\usepackage{bm}

% Some shortcuts
\newcommand\N{\ensuremath{\mathbb{N}}}
\newcommand\R{\ensuremath{\mathbb{R}}}
\newcommand\Z{\ensuremath{\mathbb{Z}}}
\renewcommand\O{\ensuremath{\emptyset}}
\newcommand\Q{\ensuremath{\mathbb{Q}}}
\newcommand\C{\ensuremath{\mathbb{C}}}

%Make implies and impliedby shorter
\let\implies\Rightarrow
\let\impliedby\Leftarrow
\let\iff\Leftrightarrow
\let\epsilon\varepsilon

% Add \contra symbol to denote contradiction
% \usepackage{stmaryrd} % for \lightning
% \newcommand\contra{\scalebox{1.5}{$\lightning$}}

% \let\phi\varphi

% Command for short corrections
% Usage: 1+1=\correct{3}{2}

\definecolor{correct}{HTML}{009900}
\newcommand\correct[2]{\ensuremath{\:}{\color{red}{#1}}\ensuremath{\to }{\color{correct}{#2}}\ensuremath{\:}}
\newcommand\green[1]{{\color{correct}{#1}}}

% horizontal rule
% \newcommand\hr{
%     \noindent\rule[0.5ex]{\linewidth}{0.5pt}
% }

% hide parts
\newcommand\hide[1]{}

% Environments
\makeatother
% For box around Definition, Theorem, \ldots
\usepackage{mdframed}
\mdfsetup{skipabove=1em,skipbelow=1em}
\theoremstyle{definition}

\newmdtheoremenv[nobreak=true]{definition}{Definition}
\newtheorem*{eg}{Example}
\newtheorem*{notation}{Notation}
\newtheorem*{previouslyseen}{As previously seen}
\newtheorem*{remark}{Remark}
\newtheorem*{note}{Note}
\newtheorem*{problem}{Problem}
\newtheorem*{observe}{Observe}
\newtheorem*{property}{Property}
\newtheorem*{intuition}{Intuition}
\newmdtheoremenv[nobreak=true]{prop}{Proposition}
\newmdtheoremenv[nobreak=true]{theorem}{Theorem}
\newmdtheoremenv[nobreak=true]{corollary}{Corollary}

% \newtcbtheorem{uctheorem}{Theorem}{uncheckedstyle}{theo}
% End example and intermezzo environments with a small diamond (just like proof
% environments end with a small square)
\usepackage{etoolbox}
\AtEndEnvironment{vb}{\null\hfill$\diamond$}%
\AtEndEnvironment{intermezzo}{\null\hfill$\diamond$}%
% \AtEndEnvironment{opmerking}{\null\hfill$\diamond$}%

% Fix some spacing
% http://tex.stackexchange.com/questions/22119/how-can-i-change-the-spacing-before-theorems-with-amsthm
\makeatletter
\def\thm@space@setup{%
	\thm@preskip=\parskip \thm@postskip=0pt
}


% Exercise 
% Usage:
% \oefening{5}
% \suboefening{1}
% \suboefening{2}
% \suboefening{3}
% gives
% Oefening 5
%   Oefening 5.1
%   Oefening 5.2
%   Oefening 5.3
\newcommand{\oefening}[1]{%
	\def\@oefening{#1}%
	\subsection*{Oefening #1}
}

\newcommand{\suboefening}[1]{%
	\subsubsection*{Oefening \@oefening.#1}
}


% \lecture starts a new lecture (les in dutch)
%
% Usage:
% \lecture{1}{di 12 feb 2019 16:00}{Inleiding}
%
% This adds a section heading with the number / title of the lecture and a
% margin paragraph with the date.

% I use \dateparts here to hide the year (2019). This way, I can easily parse
% the date of each lecture unambiguously while still having a human-friendly
% short format printed to the pdf.

% \usepackage{xifthen}
% \def\testdateparts#1{\dateparts#1\relax}
% \def\dateparts#1 #2 #3 #4 #5\relax{
% 	\marginpar{\small\textsf{\mbox{#1 #2 #3 #5}}}
% }

% \def\@lecture{}%
% \newcommand{\lecture}[3]{
% 	\ifthenelse{\isempty{#3}}{%
% 		\def\@lecture{Lecture #1}%
% 	}{%
% 		\def\@lecture{Lecture #1: #3}%
% 	}%
% 	\subsection*{\@lecture}
% 	% \marginpar{\small\textsf{\mbox{#2}}}
% }



% These are the fancy headers
\usepackage{fancyhdr}
\pagestyle{fancy}

% LE: left even
% RO: right odd
% CE, CO: center even, center odd
% My name for when I print my lecture notes to use for an open book exam.
\fancyhead[LE,RO]{Kristian Sørdal}

\fancyhead[RO,LE]{DAVE3700 - Matte 3000} % Right odd,  Left even
\fancyhead[RE,LO]{\leftmark}          % Right even, Left odd

\fancyfoot[RO,LE]{\thepage}  % Right odd,  Left even
\fancyfoot[RE,LO]{}          % Right even, Left odd
\fancyfoot[C]{\leftmark}     % Center

\makeatother

% Todonotes and inline notes in fancy boxes
\usepackage{todonotes}
\usepackage{tcolorbox}

% Make boxes breakable
\tcbuselibrary{breakable}

% Figure support as explained in my blog post.
\usepackage{import}
\usepackage{xifthen}
\usepackage{pdfpages}
\usepackage{transparent}
\newcommand{\incfig}[1]{%
	\def\svgwidth{\columnwidth}
	\import{./figures/}{#1.pdf_tex}
}

% Fix some stuff
% %http://tex.stackexchange.com/questions/76273/multiple-pdfs-with-page-group-included-in-a-single-page-warning
\pdfsuppresswarningpagegroup=1

\author{Kristian Sørdal}

\begin{document}
\section{Lecture 16}

\subsection{Change of variables (cont.)}

We have seen that the \textit{Jacobian Determinant} \( J \), and

\[ \iint_{R}fdA = \iint_{S} f\left( x\left( u,v \right), y\left( u,v \right) \right)\left| J\left( u,v \right) \right|dudv \]

The Jacobian appears since

\begin{figure}[h]
	\centering
	\incfig{changevariables}
\end{figure}

\begin{eg}
	Consider \( \left( u,v \right) \) defined by

	\[ u = ax, \quad v =by, a,b > 0 \]

	Now consider the square

	\[ R = \left[ 1,0 \right]\times \left[ 0,1 \right] \]


	\begin{figure}[H]
		\centering
		\incfig{squareuvcoords}
	\end{figure}

	It becomes the rectangle \( S=\left[ 0,a \right]\times \left[ 0,b \right] \). Let us also write

	\[ x = \frac{u}{a} , \quad y = \frac{v}{b} \]

	Then we compute the Jacobian.

	\[ J = \left| \begin{bmatrix}
			\frac{1}{a} & 0           \\
			0           & \frac{1}{b}
		\end{bmatrix} \right| =  \frac{1}{ab} \]

	We have \( \iint_{R} 1dA = 1 \), using the theorem, we compute

	\begin{align*}
		\iint_{R}1dA & = \iint_{S}\left| J\left( u,v \right) \right|dudv  \\
		             & = \int_{u=0}^{a} \int_{v=0}^{b} \frac{1}{ab} du dv \\
		             & = 1
	\end{align*}
\end{eg}

\begin{eg}
	Consider the elliptical region

	\[ R: x^2-xy + y^2 \leq 2, \quad f\left( x,y \right) = x^2-xy+y^2 \]


	\begin{figure}[H]
		\centering
		\incfig{ellipticalregion}
	\end{figure}

	We want to compute \( \iint_{R}fdA \). Consider \( \left( u,v \right) \) defined by

	\[ x = \sqrt{2} - \sqrt{\frac{2}{3} }u, \quad y = \sqrt{2} + \sqrt{\frac{2}{3} }v \]

	\[ x^2-xy + y^2 = 2u^2+2v^2 \]

	In \( \left( u,v \right) \) coordinates, we get

	\begin{align*}
		S & = \left\{ \left( u,v \right): 2u^2+2v^2 \leq 2 \right\} \\
		  & = \left\{ \left( u,v \right): u^2+v^2 \leq 1 \right\}
	\end{align*}

	This equals a circle. By the theorem, we have that

	\[ \iint_{R}fdA = \iint_{S}f\left( x\left( u,v \right),y\left( u,v \right) \right)\left| J\left( u,v \right) \right|dudv \]

	For \( f \), we have

	\[ f\left( x,y \right)=x^2-xy+y^2 = 2\left( u^2+v^2 \right) \]

	The Jacobian is

	\[ J\left( u,v \right) = \left| \begin{bmatrix}
			\sqrt{2} & -\sqrt{\frac{2}{3} } \\
			\sqrt{2} & \sqrt{\frac{2}{3} }
		\end{bmatrix} \right| = \frac{2}{\sqrt{3}} - \left( -\frac{2}{\sqrt{3}} \right) = \frac{4}{\sqrt{3}} \]

	\[ \iint_{R}fdA = \iint_{S}2\left( u^2+v^2 \right) \frac{4}{\sqrt{3}}dudv \]

	Since \( S \) is a disc, we introduce

	\[ u = r\cos\theta,\quad v = r\sin\theta \]

	The disc is then \( 0 \leq r \leq 1 \), \( 0 \leq \theta \leq 2\pi \). We obtain


	\begin{align*}
		\iint_{R}fdA & = \frac{8}{\sqrt{3}\int_{r=0}^{1} \int_{\theta=0}^{2\pi} r^2 \cdot r d\theta dr} \\
		             & =\frac{16\pi}{\sqrt{3}}                                                          \\
		             & = \int_{r=0}^{1} r^3 dr                                                          \\
		             & =\frac{4\pi}{\sqrt{3}}
	\end{align*}

\end{eg}

\begin{note}
	The aboslute value in \( \left| J\left( u,v \right) \right| \) is important!
\end{note}

\subsection{Triple Integrals}
When doing a triple integral, we proceed as with double integrals. The analogue of the rectangle are boxes. A box is a region of the following form

\[ B = \left[ x_{0} , x_{1}  \right]\times \left[ y_{0} ,y_{1}  \right] \times  \left[ z_{0} , z_{1}  \right] \]


\begin{figure}[H]
	\centering
	\incfig{box}
\end{figure}

Its volume is

\[ \text{vol}(B) = \left( x_{1} - x_{0}  \right)\left( y_{1} - y_{0}  \right)\left( z_{1} - z_{0}  \right) \]

We approximate a region with boxes \( B_{i} \) of volumes \( \Delta v_{i} \). Pick a sample point \( \left( x_{i}^{\star},y_{i}^{\star},z_{i}^{\star} \right) \) in each \( B_{i} \). Given \( f\left( x,y,z \right) \) we consider

\[ \sum_{i}^{} f\left(x_{i}^{\star},y_{i}^{\star},z_{i}^{\star}  \right)\Delta_{i} \]

in the unit of the integral.

\begin{definition}
	The triple integral of \( f \) over \( t \) is

	\[ \iiint_{T}fdV = \lim_{\Delta v_i\rightarrow 0} \sum_{i}^{} f\left(x_{i}^{\star},y_{i}^{\star},z_{i}^{\star}  \right)\Delta_{i}    \]
\end{definition}

When \( f=1 \) we get the volume of \( T \). In the case of \( f > 0 \), it is less intuitive. How do we interpret \( \int_{T} fdV \)? We want to think of \( f \) as a local density.

\begin{eg}
	Consider a 3D-object described by a 3D-region \( T \), with \( \rho\left( x,y,z \right) \) its mass density. Then its mass is given by

	\[ m = \iiint_{T}\rho dV \]
\end{eg}

\begin{prop}
	Linearity
	\medskip

	If \( a \) and \( b \) are constants, then

	\[ \iiint_{T}\left( af + bg \right)dV = a \iiint_{T}fdV + b \iiint_{T}gdV \]
\end{prop}

\begin{prop}
	Partitions
	\medskip

	Let \( T \text{ and } T' \) be non-overlapping regions, then

	\[ \iiint_{T \cup T'} fdV = \iiint_{T} fdV + \iiint_{T'} fdV \]


	\begin{figure}[H]
		\centering
		\incfig[0.7]{partitionscube}
	\end{figure}
\end{prop}

\subsection{Integration Over Boxes}

Boxes are usually the easiest regions to consider when doing a triple integral, as mentioned before.

\begin{prop}
	Let \( T = \left[ x_{0} , x_{1}  \right]\times \left[ y_{0} ,y_{1}  \right] \times  \left[ z_{0} , z_{1}  \right] \), then

	\[ \iiint_{T}fdV = \int_{z_{0} }^{z_{1} } \int_{y_{0} }^{y_{1} } \int_{x_{0} }^{x_{1} } f\left( x,y,z \right) dx dy dz \]

	The order can be exchanged.
\end{prop}

\begin{eg}
	As expected, \( f=1 \) gives the volume.


	\begin{align*}
		\iiint_{T}fdV & = \int_{z_{0} }^{z_{1} } \int_{y_{0} }^{y_{1} } \int_{x_{0} }^{x_{1} } f\left( x,y,z \right) dx dy dz \\
		              & = \left( x_{1} - x_{0}  \right)\int_{z_{0} }^{z_{1} } \int_{y_{0} }^{y_{1} } 1 dy dz                  \\
		              & = \left( x_{1} - x_{0}  \right)\left( y_{1} -y_{0}  \right)\left( z_{1} -z_{0}  \right)               \\
	\end{align*}

\end{eg}

\begin{eg}
	Consider \( T = \left[ 0,1 \right]\times \left[ 0,1 \right]\times \left[ -1,1 \right] \) and \( f\left( x,y,z \right)=2 \). Compute \( \int_{T}fdV = \int_{T}2dV\). We have

	\begin{align*}
		\iiint_{T}fdV & = \int_{z=-1}^{1} \int_{y=0}^{1} \int_{x=0}^{1} z dx dy dz \\
		              & = \int_{z=-1}^{1} \int_{y=0}^{1} z dy dz                   \\
		              & = \left[ \frac{1}{2} z^2  \right]_{-1}^{1}                 \\
		              & = \frac{1}{2} - \frac{1}{2}                                \\
		              & = 0
	\end{align*}

	Interpretation: the contributions of \( f \) cancel over \( -1 \leq z \leq 1 \), notice that

	\[ f\left( x,y,z \right) = -f\left( x,y,z \right) \]

	and \( \left[ -1,1 \right] \) is unchanged under \( z \rightarrow -2 \).
\end{eg}

\subsection{General Regions}
This is an analoguq of \( x \) and \( y \)-simple for double integrals. Suppose that we have

\[ T = \left\{ \left( x,y,z \right): \left( x,y \right)\in D, \quad u_{1} \left( x,y \right) \leq z \leq u_{2} \left( x,y \right) \right\} \]


\begin{figure}[H]
	\centering
	\incfig{doubleintsimple}
\end{figure}

We can interpret \( D \) as the projection of T in the \( xy- \)plane. It is also the domain og \( u_{1} \text{ and }u_{2}  \).

\begin{prop}
	Suppose \( T \) is a region as above, then

	\[ \iiint_{T} fdV = \iint_{D}\left[ \int_{u_{1} \left( x,y \right)}^{u_{2} \left( x,y \right)} f\left( x,y,z \right) dz \right]dA \]
\end{prop}

\begin{eg}
	Suppose \( T \) is of the form

	\[ T = \left\{ \left( x,y,z \right): \left( x,y \right) \in D, \quad 0 \leq x \leq f\left( x,y \right) \right\} \]

	Then we have

	\begin{align*}
		\iiint_{T} 1 dV & = \iint_{D}\left[\int_{0}^{f\left( x,y \right)}  dx 1dz \right]dA \\
		                & = \iint_{D}f\left( x,y \right)dA
	\end{align*}
\end{eg}

We get the double integral of \( f \) over \( D \), or the volume under the surface \( f\left( x,y \right) \). More generally, consider

\[ u_{1} \left( x,y \right) \leq z \leq u_{2} \left( x,y \right) \]

Then we have

\begin{align*}
	\iiint_{T}1dV & = \iint_{D}\left( u_{2} - u_{1}  \right)dA \\
\end{align*}

Which can be read as "volume under \( u_{2}  \)" - "volume under \( u_{1}  \)".

\[  \]
\end{document}
