\documentclass{article}
% Some basic packages
\usepackage[utf8]{inputenc}
\usepackage[margin=1.2in]{geometry}
\usepackage{textcomp}
\usepackage{url}
\usepackage{graphicx}
\usepackage{float}
\usepackage{enumitem}
\usepackage{standalone}
\usepackage{tcolorbox}
\usepackage{wrapfig}

%color settings
\usepackage{xcolor}
\definecolor{gruvbgdark}{HTML}{1d2021}
\definecolor{gruvtextdark}{HTML}{ebdbb2}
\definecolor{gruvbglight}{HTML}{f9f5d7}
\definecolor{gruvtextlight}{HTML}{3c3836}
% \pagecolor{gruvbgdark}
% \color{gruvtextdark}

% Hide page number when page is empty
\usepackage{emptypage}
\usepackage{subcaption}
\usepackage{multicol}
\usepackage{xcolor}

% Math stuff
\usepackage{amsmath, amsfonts, mathtools, amsthm, amssymb}
% Fancy script capitals
\usepackage{mathrsfs}
\usepackage{cancel}

% Bold math
\usepackage{bm}

% Some shortcuts
\newcommand\N{\ensuremath{\mathbb{N}}}
\newcommand\R{\ensuremath{\mathbb{R}}}
\newcommand\Z{\ensuremath{\mathbb{Z}}}
\renewcommand\O{\ensuremath{\emptyset}}
\newcommand\Q{\ensuremath{\mathbb{Q}}}
\newcommand\C{\ensuremath{\mathbb{C}}}

%Make implies and impliedby shorter
\let\implies\Rightarrow
\let\impliedby\Leftarrow
\let\iff\Leftrightarrow
\let\epsilon\varepsilon

% Add \contra symbol to denote contradiction
% \usepackage{stmaryrd} % for \lightning
% \newcommand\contra{\scalebox{1.5}{$\lightning$}}

% \let\phi\varphi

% Command for short corrections
% Usage: 1+1=\correct{3}{2}

\definecolor{correct}{HTML}{009900}
\newcommand\correct[2]{\ensuremath{\:}{\color{red}{#1}}\ensuremath{\to }{\color{correct}{#2}}\ensuremath{\:}}
\newcommand\green[1]{{\color{correct}{#1}}}

% horizontal rule
% \newcommand\hr{
%     \noindent\rule[0.5ex]{\linewidth}{0.5pt}
% }

% hide parts
\newcommand\hide[1]{}

% Environments
\makeatother
% For box around Definition, Theorem, \ldots
\usepackage{mdframed}
\mdfsetup{skipabove=1em,skipbelow=1em}
\theoremstyle{definition}

\newmdtheoremenv[nobreak=true]{definition}{Definition}
\newtheorem*{eg}{Example}
\newtheorem*{notation}{Notation}
\newtheorem*{previouslyseen}{As previously seen}
\newtheorem*{remark}{Remark}
\newtheorem*{note}{Note}
\newtheorem*{problem}{Problem}
\newtheorem*{observe}{Observe}
\newtheorem*{property}{Property}
\newtheorem*{intuition}{Intuition}
\newmdtheoremenv[nobreak=true]{prop}{Proposition}
\newmdtheoremenv[nobreak=true]{theorem}{Theorem}
\newmdtheoremenv[nobreak=true]{corollary}{Corollary}

% \newtcbtheorem{uctheorem}{Theorem}{uncheckedstyle}{theo}
% End example and intermezzo environments with a small diamond (just like proof
% environments end with a small square)
\usepackage{etoolbox}
\AtEndEnvironment{vb}{\null\hfill$\diamond$}%
\AtEndEnvironment{intermezzo}{\null\hfill$\diamond$}%
% \AtEndEnvironment{opmerking}{\null\hfill$\diamond$}%

% Fix some spacing
% http://tex.stackexchange.com/questions/22119/how-can-i-change-the-spacing-before-theorems-with-amsthm
\makeatletter
\def\thm@space@setup{%
	\thm@preskip=\parskip \thm@postskip=0pt
}


% Exercise 
% Usage:
% \oefening{5}
% \suboefening{1}
% \suboefening{2}
% \suboefening{3}
% gives
% Oefening 5
%   Oefening 5.1
%   Oefening 5.2
%   Oefening 5.3
\newcommand{\oefening}[1]{%
	\def\@oefening{#1}%
	\subsection*{Oefening #1}
}

\newcommand{\suboefening}[1]{%
	\subsubsection*{Oefening \@oefening.#1}
}


% \lecture starts a new lecture (les in dutch)
%
% Usage:
% \lecture{1}{di 12 feb 2019 16:00}{Inleiding}
%
% This adds a section heading with the number / title of the lecture and a
% margin paragraph with the date.

% I use \dateparts here to hide the year (2019). This way, I can easily parse
% the date of each lecture unambiguously while still having a human-friendly
% short format printed to the pdf.

% \usepackage{xifthen}
% \def\testdateparts#1{\dateparts#1\relax}
% \def\dateparts#1 #2 #3 #4 #5\relax{
% 	\marginpar{\small\textsf{\mbox{#1 #2 #3 #5}}}
% }

% \def\@lecture{}%
% \newcommand{\lecture}[3]{
% 	\ifthenelse{\isempty{#3}}{%
% 		\def\@lecture{Lecture #1}%
% 	}{%
% 		\def\@lecture{Lecture #1: #3}%
% 	}%
% 	\subsection*{\@lecture}
% 	% \marginpar{\small\textsf{\mbox{#2}}}
% }



% These are the fancy headers
\usepackage{fancyhdr}
\pagestyle{fancy}

% LE: left even
% RO: right odd
% CE, CO: center even, center odd
% My name for when I print my lecture notes to use for an open book exam.
\fancyhead[LE,RO]{Kristian Sørdal}

\fancyhead[RO,LE]{DAVE3700 - Matte 3000} % Right odd,  Left even
\fancyhead[RE,LO]{\leftmark}          % Right even, Left odd

\fancyfoot[RO,LE]{\thepage}  % Right odd,  Left even
\fancyfoot[RE,LO]{}          % Right even, Left odd
\fancyfoot[C]{\leftmark}     % Center

\makeatother

% Todonotes and inline notes in fancy boxes
\usepackage{todonotes}
\usepackage{tcolorbox}

% Make boxes breakable
\tcbuselibrary{breakable}

% Figure support as explained in my blog post.
\usepackage{import}
\usepackage{xifthen}
\usepackage{pdfpages}
\usepackage{transparent}
\newcommand{\incfig}[1]{%
	\def\svgwidth{\columnwidth}
	\import{./figures/}{#1.pdf_tex}
}

% Fix some stuff
% %http://tex.stackexchange.com/questions/76273/multiple-pdfs-with-page-group-included-in-a-single-page-warning
\pdfsuppresswarningpagegroup=1

\author{Kristian Sørdal}

\begin{document}
\section{Lecture 11}

\subsection{Operations on fields}

\begin{eg}
	Consider \( \vec{F}\left( x,y,z \right) = \left( x^2,y^2,z^2 \right) \), that is

	\[ P =x^2, \quad Q = y^2, \quad R=z^2 \]

	Its divergence is

	\[ \nabla \cdot  \vec{F} = \frac{\partial P}{\partial x}+\frac{\partial Q}{\partial y}+\frac{\partial R}{\partial z} = 2x+2y+2z \]

	We can check that \( \vec{F} \) is a gradient field. We have that \( \vec{F}=\nabla f \), with the potential

	\[ f\left( x,y,z \right)=\frac{1}{3} \left( x^2+y^2+z^2 \right)  \]
\end{eg}

\( \nabla \times \vec{F} \) is computed as a determinant. We can use the cofactor, or the Laplace expansion.

\[ \nabla \times \vec{F} =
	\left|\begin{bmatrix}
		\vec{i}    & \vec{j}    & \vec{k}    \\
		\partial_x & \partial_y & \partial_z \\
		P          & Q          & R
	\end{bmatrix}\right| = \begin{bmatrix}
		\partial_{y} & \partial_{z} \\
		Q            & R
	\end{bmatrix}\vec{i} - \begin{bmatrix}
		\partial_{x} & \partial_{z} \\
		P            & R
	\end{bmatrix}\vec{j} + \begin{bmatrix}
		\partial_{x} & \partial_{y} \\
		P            & Q
	\end{bmatrix}\vec{k}\]

\[= \vec{i}\left(\partial_y R - \partial_z Q \right) -\vec{j}\left(\partial_x R - \partial_z P \right) + \vec{k}\left(\partial_x - Q \partial_y P \right)\]

This is a concrete formula for \( \nabla \times \vec{F} \).

\begin{eg}
	Consider \( \vec{F} = xy \vec{i} + \left( x+z \right)\vec{j} + yz \vec{k} \). We compute

	\[ \nabla \times  \vec{F} =  \vec{i}\left(\partial_y R - \partial_z Q \right) -\vec{j}\left(\partial_x R - \partial_z P \right) + \vec{k}\left(\partial_x - Q \partial_y P \right) \]

	where

	\begin{align*}
		 & P = xy, & Q =x+z, &  & R = yz
	\end{align*}

	\[ \nabla \times \vec{F} = \left( 2-1 \right)\vec{i} + \left( 1-x \right)\vec{k} \]
\end{eg}

\begin{note}
	Gradient and divergence can be defined in any dimension. The curl is only defined in up to three dimensions.
\end{note}

\subsection{Intepretation of Divergence}

Think of \( \vec{F} \) as the velocity of a fluid. \( \nabla \cdot \vec{F} \) at a point \( P \) is the amount of fluid entering / leaving a small region around \( P \).

\begin{figure}[H]
	\centering
	\def\svgwidth{1\textwidth}
	\import{./figures/}{fig31.pdf_tex}
\end{figure}

\begin{eg}
	Consider \( \vec{F_{1}}\left( x,y \right)= \left( x,y \right) \), then

	\[ \nabla \cdot  \vec{F_{1}} = \frac{\partial x}{\partial x} + \frac{\partial y}{\partial y}+2 \]

	This corresponds to fluid leaving the region. Similarily for \( \vec{F_{2}} = \left( -x,-y \right) \), then

	\[ \nabla \cdot \vec{F}_{2} = \frac{\partial \left( -x \right)}{\partial x} + \frac{\partial \left( -y \right)}{\partial y}=-2 \]

	This corresponds to fluid entering the region. Finally consider \( \vec{F}_{3}=\left( 0,1 \right) \), then

	\[ \nabla \cdot \vec{F}_{3} =0 \]

	This is an equillibrium situation.
\end{eg}

\subsection{Interpretation of Curl}

\( \nabla \times  \vec{F} \) measures the "rotation" of \( \vec{F} \).

\begin{eg}
	Consider \( \vec{F}\left( x,y,z \right)=\left( x^2,0,0 \right) \). We expect no rotation, we compute

	\[ \nabla \times \vec{F} =
		\left|\begin{bmatrix}
			\vec{i}    & \vec{j}    & \vec{k}    \\
			\partial_x & \partial_y & \partial_z \\
			x^2        & 0          & 0
		\end{bmatrix}\right| = 0 \]

	\begin{figure}[H]
		\centering
		\def\svgwidth{0.5\textwidth}
		\import{./figures/}{fig32.pdf_tex}
	\end{figure}
\end{eg}

\begin{eg}
	Consider \( \vec{F}\left( x,y,z \right)=\left( -\omega y, \omega x, 0 \right) \), where \( \omega \) is a non-zero constant. We compute

	\[ \nabla \times \vec{F} =
		\left|\begin{bmatrix}
			\vec{i}    & \vec{j}    & \vec{k}    \\
			\partial_x & \partial_y & \partial_z \\
			-\omega y  & \omega x   & 0
		\end{bmatrix}\right| = \begin{bmatrix}
			\partial_{x} & \partial_{y} \\
			-\omega y    & -\omega x
		\end{bmatrix}\vec{k} = 2\omega \vec{k} \]

	\( \nabla \times \vec{F} \neq 0 \) gives non-zero rotation. Also \( \nabla \times \vec{F} = \text{ const} \) gives that that there is some rotation everywhere.

	For a physical intepretation of this, we can write

	\[ \vec{v}=\vec{F}, \quad \vec{\omega}=\left( 0,0,\omega \right), \quad \vec{r}=\left( x,y,z \right) \]

	Then we can check that \( \vec{v}=\vec{\omega}\cdot \vec{r} \). This is the velocity corresponding to the angular velocity \( \vec{\omega} \)

\end{eg}

\subsection{Scalar Field from Gradient}

Suppose that we know \(\nabla f \). Can we recover \( f \)? Yes, up to the initial conditions. The strategy is the following, first we write

\[ \vec{F} = \nabla f = \left( P,Q,R \right)\]

By the definition of the gradient field, we have

\[ \nabla f = \left( \frac{\partial f}{\partial x},\frac{\partial f}{\partial y},\frac{\partial f}{\partial z} \right) \]

Comparing, we get

\[ \frac{\partial f}{\partial x}=P, \quad \frac{\partial f}{\partial y}=Q,\quad,\quad \frac{\partial f}{\partial x}=R \]

\begin{eg}
	Suppose we are given the gradient field

	\[ \vec{F}=\nabla f = \left( 2x+2y, \frac{1}{2} x^2+3y \right) \]

	We want to find \( f \), we must have

	\[ \frac{\partial f}{\partial x}=2x+2y,\quad \frac{\partial f}{\partial y}=\frac{1}{2} x^2+3y \]

	Integrating the first equation in \( x \) gives us


	\begin{align*}
		f\left( x,y \right) & =\int\left( 2x+xy \right)dx                \\
		                    & = x^2 + \frac{1}{2} x^2y+g\left( y \right)
	\end{align*}

	Where \( g\left( y \right) \) is the integration constant. It can depend on \( y \).

	Now we compute \( \frac{\partial f}{\partial y} \). We get

	\[ \frac{\partial f}{\partial y}=\frac{1}{2} x^2 + \frac{d g}{dy} \]

	But we also have that \( \frac{\partial f}{\partial y}=\frac{1}{2} x^2 + 3y \).

	Comparing them, we get

	\[ \cancel{\frac{1}{2} x^2} + \frac{d g}{dy} = \cancel{\frac{1}{2} x^2} + 3y \]

	Then \( \frac{d g}{dy} = 3y \). Integrating in \( y \) we get

	\[ g\left( y \right)=\int 3ydy = \frac{3}{2} y^3 + C \]

	Here, \( C \) is a constant. Finally, inserting it, we get

	\begin{align*}
		f\left( x,y \right) & =x^2+\frac{1}{2} x^2 y + g\left( k \right) \\
		                    & = x^2+\frac{1}{2} x^2y + \frac{3}{2} y^3+C
	\end{align*}

	The constant \( C \) is usually not important. It can be fixed by an inital condition. For instance \( f\left( 0,0 \right) \) implies that \( C =0 \)



\end{eg}

\end{document}
