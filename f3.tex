\documentclass{article}
% Some basic packages
\usepackage[utf8]{inputenc}
\usepackage[margin=1.2in]{geometry}
\usepackage{textcomp}
\usepackage{url}
\usepackage{graphicx}
\usepackage{float}
\usepackage{enumitem}
\usepackage{standalone}
\usepackage{tcolorbox}
\usepackage{wrapfig}

%color settings
\usepackage{xcolor}
\definecolor{gruvbgdark}{HTML}{1d2021}
\definecolor{gruvtextdark}{HTML}{ebdbb2}
\definecolor{gruvbglight}{HTML}{f9f5d7}
\definecolor{gruvtextlight}{HTML}{3c3836}
% \pagecolor{gruvbgdark}
% \color{gruvtextdark}

% Hide page number when page is empty
\usepackage{emptypage}
\usepackage{subcaption}
\usepackage{multicol}
\usepackage{xcolor}

% Math stuff
\usepackage{amsmath, amsfonts, mathtools, amsthm, amssymb}
% Fancy script capitals
\usepackage{mathrsfs}
\usepackage{cancel}

% Bold math
\usepackage{bm}

% Some shortcuts
\newcommand\N{\ensuremath{\mathbb{N}}}
\newcommand\R{\ensuremath{\mathbb{R}}}
\newcommand\Z{\ensuremath{\mathbb{Z}}}
\renewcommand\O{\ensuremath{\emptyset}}
\newcommand\Q{\ensuremath{\mathbb{Q}}}
\newcommand\C{\ensuremath{\mathbb{C}}}

%Make implies and impliedby shorter
\let\implies\Rightarrow
\let\impliedby\Leftarrow
\let\iff\Leftrightarrow
\let\epsilon\varepsilon

% Add \contra symbol to denote contradiction
% \usepackage{stmaryrd} % for \lightning
% \newcommand\contra{\scalebox{1.5}{$\lightning$}}

% \let\phi\varphi

% Command for short corrections
% Usage: 1+1=\correct{3}{2}

\definecolor{correct}{HTML}{009900}
\newcommand\correct[2]{\ensuremath{\:}{\color{red}{#1}}\ensuremath{\to }{\color{correct}{#2}}\ensuremath{\:}}
\newcommand\green[1]{{\color{correct}{#1}}}

% horizontal rule
% \newcommand\hr{
%     \noindent\rule[0.5ex]{\linewidth}{0.5pt}
% }

% hide parts
\newcommand\hide[1]{}

% Environments
\makeatother
% For box around Definition, Theorem, \ldots
\usepackage{mdframed}
\mdfsetup{skipabove=1em,skipbelow=1em}
\theoremstyle{definition}

\newmdtheoremenv[nobreak=true]{definition}{Definition}
\newtheorem*{eg}{Example}
\newtheorem*{notation}{Notation}
\newtheorem*{previouslyseen}{As previously seen}
\newtheorem*{remark}{Remark}
\newtheorem*{note}{Note}
\newtheorem*{problem}{Problem}
\newtheorem*{observe}{Observe}
\newtheorem*{property}{Property}
\newtheorem*{intuition}{Intuition}
\newmdtheoremenv[nobreak=true]{prop}{Proposition}
\newmdtheoremenv[nobreak=true]{theorem}{Theorem}
\newmdtheoremenv[nobreak=true]{corollary}{Corollary}

% \newtcbtheorem{uctheorem}{Theorem}{uncheckedstyle}{theo}
% End example and intermezzo environments with a small diamond (just like proof
% environments end with a small square)
\usepackage{etoolbox}
\AtEndEnvironment{vb}{\null\hfill$\diamond$}%
\AtEndEnvironment{intermezzo}{\null\hfill$\diamond$}%
% \AtEndEnvironment{opmerking}{\null\hfill$\diamond$}%

% Fix some spacing
% http://tex.stackexchange.com/questions/22119/how-can-i-change-the-spacing-before-theorems-with-amsthm
\makeatletter
\def\thm@space@setup{%
	\thm@preskip=\parskip \thm@postskip=0pt
}


% Exercise 
% Usage:
% \oefening{5}
% \suboefening{1}
% \suboefening{2}
% \suboefening{3}
% gives
% Oefening 5
%   Oefening 5.1
%   Oefening 5.2
%   Oefening 5.3
\newcommand{\oefening}[1]{%
	\def\@oefening{#1}%
	\subsection*{Oefening #1}
}

\newcommand{\suboefening}[1]{%
	\subsubsection*{Oefening \@oefening.#1}
}


% \lecture starts a new lecture (les in dutch)
%
% Usage:
% \lecture{1}{di 12 feb 2019 16:00}{Inleiding}
%
% This adds a section heading with the number / title of the lecture and a
% margin paragraph with the date.

% I use \dateparts here to hide the year (2019). This way, I can easily parse
% the date of each lecture unambiguously while still having a human-friendly
% short format printed to the pdf.

% \usepackage{xifthen}
% \def\testdateparts#1{\dateparts#1\relax}
% \def\dateparts#1 #2 #3 #4 #5\relax{
% 	\marginpar{\small\textsf{\mbox{#1 #2 #3 #5}}}
% }

% \def\@lecture{}%
% \newcommand{\lecture}[3]{
% 	\ifthenelse{\isempty{#3}}{%
% 		\def\@lecture{Lecture #1}%
% 	}{%
% 		\def\@lecture{Lecture #1: #3}%
% 	}%
% 	\subsection*{\@lecture}
% 	% \marginpar{\small\textsf{\mbox{#2}}}
% }



% These are the fancy headers
\usepackage{fancyhdr}
\pagestyle{fancy}

% LE: left even
% RO: right odd
% CE, CO: center even, center odd
% My name for when I print my lecture notes to use for an open book exam.
\fancyhead[LE,RO]{Kristian Sørdal}

\fancyhead[RO,LE]{DAVE3700 - Matte 3000} % Right odd,  Left even
\fancyhead[RE,LO]{\leftmark}          % Right even, Left odd

\fancyfoot[RO,LE]{\thepage}  % Right odd,  Left even
\fancyfoot[RE,LO]{}          % Right even, Left odd
\fancyfoot[C]{\leftmark}     % Center

\makeatother

% Todonotes and inline notes in fancy boxes
\usepackage{todonotes}
\usepackage{tcolorbox}

% Make boxes breakable
\tcbuselibrary{breakable}

% Figure support as explained in my blog post.
\usepackage{import}
\usepackage{xifthen}
\usepackage{pdfpages}
\usepackage{transparent}
\newcommand{\incfig}[1]{%
	\def\svgwidth{\columnwidth}
	\import{./figures/}{#1.pdf_tex}
}

% Fix some stuff
% %http://tex.stackexchange.com/questions/76273/multiple-pdfs-with-page-group-included-in-a-single-page-warning
\pdfsuppresswarningpagegroup=1

\author{Kristian Sørdal}

\begin{document}
\section{Derivatives}

\subsection{Partial Derivatives}

In the case of one variable, we have

\[ \frac{d f}{dt} = \lim_{n\rightarrow 0} \frac{f\left( x+n \right) - f(x)}{n} \]

Similarily, for two of more variables, we have the following definition

\begin{definition}
	The partial derivative of \( f\left( x,y \right) \) with respect to x:

	\[ \frac{\partial f}{\partial x} = \lim_{n\rightarrow 0} \frac{f\left( x + n,y \right) - f\left( x,y \right)}{n} \]

	Also written as \( f_{x} \), for \( \frac{\partial f}{\partial y} \), we have \( f_{y} \)
\end{definition}

Note, the expression above is \( \frac{\partial f}{\partial x}(x,y) \), which is the value at the point \( \left( x,y \right) \)

\subsection{Higher order derivatives}

Given \( \frac{\partial f}{\partial x} \), we can take further derivatives. We have

\[ \frac{\partial^2 f }{\partial^2 x} =\frac{\partial }{\partial x} \left( \frac{\partial f}{\partial x} \right),\quad \frac{\partial^2 f }{\partial x\partial y} = \frac{\partial }{\partial y}\left( \frac{\partial f}{\partial x} \right)\]

Also written as \( f_{xx}, f_{yy}, f_{xy}, f_{yx} \). In most cases, \( f_{xy} \text{ and } f_{yx} \) coincide.

\begin{theorem}
	Scwartz theorem: Suppose \( f_{xy} \text{ and } f_{yx} \) exist, and are continous, then

	\[ f_{xy} = f_{yx} \]
\end{theorem}

Similar definitions and results for the case of more variables: \( x_{1} , \ldots , x_{n} \), with \( n \) variables.

\subsection{Chain Rule}
Suppose \( f(x) = g(h(x)) \), for instance

\[ f(x) = \left( \cos x \right)^2 \text{ with } g(x)=x^2, h(x)=\cos x \]

then the chain rule is

\[ \frac{d f}{dt}\left( x_{0}  \right) = \frac{d g}{dh}\left( h\left( x_{0}  \right) \right) \cdot \frac{d h}{dt}\left( x_{0}  \right) \]

Generalization to more variables.

\begin{theorem}
	Chain rule: consider \( f\left( x,y \right) \) \( x \text{ and } y \) depending on a variable \( t \). Then:

	\[ \frac{d f}{dt}t_{0} = \frac{\partial f}{\partial x}\left( x\left( t_{0}  \right), y\left( t_{0}  \right) \right)\frac{d x}{dt}t_{0} + \frac{\partial f}{\partial y}\left( x\left( t_{0}  \right), y\left( t_{0}  \right) \right)\frac{d y}{dt}t_{0}  \]
\end{theorem}

The "short form" of this result is

\[ \frac{d f}{dt} = \frac{\partial f}{\partial x}\frac{d x}{dt} + \frac{\partial f}{\partial y}\frac{d y}{dt} \]

\begin{eg}
	Consiter \( f\left( x,y \right) = xy \), where

	\[ x(t) = \cos t, y(t) = \sin t\]

	This cannot be computed directly with \( \frac{d f}{dt} \).

	\[ f(t)=f\left( x\left( t \right),y\left( t \right) \right) = f\left( \cos t, \sin t \right) = \cos t \cdot  \sin t \]

	We can compute

	\[ \frac{d f}{dt}=\left( \cos t \right)' \sin t + \cos t(\sin t)' = -\left( \sin t \right)^2 + \left( \cos t \right)^2\]

	Usin the chain rule, we get
	\begin{align*}
		\frac{d f}{dt} & j= \frac{\partial f}{\partial x}\frac{d x}{dt} + \frac{\partial f}{\partial y}\frac{d y}{dt} \\
	\end{align*}
\end{eg}

\subsection{The Gradient}
Define an operation that takes a scalar function, and returns a vector function.

\begin{definition}
	The gradient of \( f\left( x,y \right) \text{ at }\left( x_{0} , y_{0}  \right) \) is

	\[ \nabla f\left( x_{0} , y_{0}  \right) = \left( f_{x}\left( x_{0} , y_{0}  \right), f_{y}(x_{0} , y_{0} ) \right) \]
\end{definition}
\end{document}
