\documentclass{article}
% Some basic packages
\usepackage[utf8]{inputenc}
\usepackage[margin=1.2in]{geometry}
\usepackage{textcomp}
\usepackage{url}
\usepackage{graphicx}
\usepackage{float}
\usepackage{enumitem}
\usepackage{standalone}
\usepackage{tcolorbox}
\usepackage{wrapfig}

%color settings
\usepackage{xcolor}
\definecolor{gruvbgdark}{HTML}{1d2021}
\definecolor{gruvtextdark}{HTML}{ebdbb2}
\definecolor{gruvbglight}{HTML}{f9f5d7}
\definecolor{gruvtextlight}{HTML}{3c3836}
% \pagecolor{gruvbgdark}
% \color{gruvtextdark}

% Hide page number when page is empty
\usepackage{emptypage}
\usepackage{subcaption}
\usepackage{multicol}
\usepackage{xcolor}

% Math stuff
\usepackage{amsmath, amsfonts, mathtools, amsthm, amssymb}
% Fancy script capitals
\usepackage{mathrsfs}
\usepackage{cancel}

% Bold math
\usepackage{bm}

% Some shortcuts
\newcommand\N{\ensuremath{\mathbb{N}}}
\newcommand\R{\ensuremath{\mathbb{R}}}
\newcommand\Z{\ensuremath{\mathbb{Z}}}
\renewcommand\O{\ensuremath{\emptyset}}
\newcommand\Q{\ensuremath{\mathbb{Q}}}
\newcommand\C{\ensuremath{\mathbb{C}}}

%Make implies and impliedby shorter
\let\implies\Rightarrow
\let\impliedby\Leftarrow
\let\iff\Leftrightarrow
\let\epsilon\varepsilon

% Add \contra symbol to denote contradiction
% \usepackage{stmaryrd} % for \lightning
% \newcommand\contra{\scalebox{1.5}{$\lightning$}}

% \let\phi\varphi

% Command for short corrections
% Usage: 1+1=\correct{3}{2}

\definecolor{correct}{HTML}{009900}
\newcommand\correct[2]{\ensuremath{\:}{\color{red}{#1}}\ensuremath{\to }{\color{correct}{#2}}\ensuremath{\:}}
\newcommand\green[1]{{\color{correct}{#1}}}

% horizontal rule
% \newcommand\hr{
%     \noindent\rule[0.5ex]{\linewidth}{0.5pt}
% }

% hide parts
\newcommand\hide[1]{}

% Environments
\makeatother
% For box around Definition, Theorem, \ldots
\usepackage{mdframed}
\mdfsetup{skipabove=1em,skipbelow=1em}
\theoremstyle{definition}

\newmdtheoremenv[nobreak=true]{definition}{Definition}
\newtheorem*{eg}{Example}
\newtheorem*{notation}{Notation}
\newtheorem*{previouslyseen}{As previously seen}
\newtheorem*{remark}{Remark}
\newtheorem*{note}{Note}
\newtheorem*{problem}{Problem}
\newtheorem*{observe}{Observe}
\newtheorem*{property}{Property}
\newtheorem*{intuition}{Intuition}
\newmdtheoremenv[nobreak=true]{prop}{Proposition}
\newmdtheoremenv[nobreak=true]{theorem}{Theorem}
\newmdtheoremenv[nobreak=true]{corollary}{Corollary}

% \newtcbtheorem{uctheorem}{Theorem}{uncheckedstyle}{theo}
% End example and intermezzo environments with a small diamond (just like proof
% environments end with a small square)
\usepackage{etoolbox}
\AtEndEnvironment{vb}{\null\hfill$\diamond$}%
\AtEndEnvironment{intermezzo}{\null\hfill$\diamond$}%
% \AtEndEnvironment{opmerking}{\null\hfill$\diamond$}%

% Fix some spacing
% http://tex.stackexchange.com/questions/22119/how-can-i-change-the-spacing-before-theorems-with-amsthm
\makeatletter
\def\thm@space@setup{%
	\thm@preskip=\parskip \thm@postskip=0pt
}


% Exercise 
% Usage:
% \oefening{5}
% \suboefening{1}
% \suboefening{2}
% \suboefening{3}
% gives
% Oefening 5
%   Oefening 5.1
%   Oefening 5.2
%   Oefening 5.3
\newcommand{\oefening}[1]{%
	\def\@oefening{#1}%
	\subsection*{Oefening #1}
}

\newcommand{\suboefening}[1]{%
	\subsubsection*{Oefening \@oefening.#1}
}


% \lecture starts a new lecture (les in dutch)
%
% Usage:
% \lecture{1}{di 12 feb 2019 16:00}{Inleiding}
%
% This adds a section heading with the number / title of the lecture and a
% margin paragraph with the date.

% I use \dateparts here to hide the year (2019). This way, I can easily parse
% the date of each lecture unambiguously while still having a human-friendly
% short format printed to the pdf.

% \usepackage{xifthen}
% \def\testdateparts#1{\dateparts#1\relax}
% \def\dateparts#1 #2 #3 #4 #5\relax{
% 	\marginpar{\small\textsf{\mbox{#1 #2 #3 #5}}}
% }

% \def\@lecture{}%
% \newcommand{\lecture}[3]{
% 	\ifthenelse{\isempty{#3}}{%
% 		\def\@lecture{Lecture #1}%
% 	}{%
% 		\def\@lecture{Lecture #1: #3}%
% 	}%
% 	\subsection*{\@lecture}
% 	% \marginpar{\small\textsf{\mbox{#2}}}
% }



% These are the fancy headers
\usepackage{fancyhdr}
\pagestyle{fancy}

% LE: left even
% RO: right odd
% CE, CO: center even, center odd
% My name for when I print my lecture notes to use for an open book exam.
\fancyhead[LE,RO]{Kristian Sørdal}

\fancyhead[RO,LE]{DAVE3700 - Matte 3000} % Right odd,  Left even
\fancyhead[RE,LO]{\leftmark}          % Right even, Left odd

\fancyfoot[RO,LE]{\thepage}  % Right odd,  Left even
\fancyfoot[RE,LO]{}          % Right even, Left odd
\fancyfoot[C]{\leftmark}     % Center

\makeatother

% Todonotes and inline notes in fancy boxes
\usepackage{todonotes}
\usepackage{tcolorbox}

% Make boxes breakable
\tcbuselibrary{breakable}

% Figure support as explained in my blog post.
\usepackage{import}
\usepackage{xifthen}
\usepackage{pdfpages}
\usepackage{transparent}
\newcommand{\incfig}[1]{%
	\def\svgwidth{\columnwidth}
	\import{./figures/}{#1.pdf_tex}
}

% Fix some stuff
% %http://tex.stackexchange.com/questions/76273/multiple-pdfs-with-page-group-included-in-a-single-page-warning
\pdfsuppresswarningpagegroup=1

\author{Kristian Sørdal}

\begin{document}
\section{Lecture 15}
\subsection{General Regions (cont.)}
As we saw in the previous lecture, we have the following regions

\begin{figure}[H]
	\centering
	\incfig{ysimple}
	\caption{y-simple}
\end{figure}

\begin{figure}[H]
	\centering
	\incfig{xsimple}
	\caption{x-simple}
\end{figure}

\begin{eg}
	Consider the region with \( -1 \leq x \leq 1 \) and bounder by

	\[ a\left( y \right) = y^2, \quad b\left( y \right)=y^2+\frac{1}{2}  \]


	\begin{figure}[H]
		\centering
		\incfig{examplexsimple}
	\end{figure}

	This is an x-simple region.
\end{eg}

More general regions can be partitioned into x-simple and y-simple regions. A region can be described in many different ways.

\begin{eg}
	Consider the region

	\[ D =\left\{ \left( x,y \right): 0 \leq x \leq 1, \quad -x \leq y \leq x\right\} \]
	\begin{figure}[H]
		\centering
		\incfig{ysimpletriangle}
	\end{figure}
	This is a y-simple region. We can also describe this triangle using x-simple regions.


	\begin{figure}[H]
		\centering
		\incfig{xsimpletriangle}
	\end{figure}

	We have two regions:

	\begin{align*}
		 & D_{1} = \left\{ \left( x,y \right): y \leq x \leq 1, \quad 0 \leq y \leq 1 \right\}   \\
		 & D_{2} = \left\{ \left( x,y \right): -y \leq x \leq 1, \quad -1 \leq y \leq 0 \right\}
	\end{align*}

	We have that \( D = D_{1} \cup D_{2}  \), and

	\[ \iint_{D}fdA = \iint_{D_{1} }fdA + \iint_{D_{2} } fdA \]
\end{eg}

\subsection{Integrations in polar coordinates}

We want to compute \( \iint_{D}fdA  \) using the polar coordinates \( \left( r, \theta \right) \), recall that

\[ x = r \cos \theta, \quad y = r \sin \theta \]

In cartesian coordinates, we use the "infinitesimal area \( dA = dxdy \)". We will now consider small polar regions.


\begin{figure}[H]
	\centering
	\incfig{polarregions}
\end{figure}

Recall that area is equal to \( \frac{1}{2} \theta r^2 \).

\begin{align*}
	\Delta A & = \text{"Large region"} - \text{"Small Region"}                                           \\
	         & = \frac{1}{2} \Delta \theta \left( r + \Delta r \right)^2 - \frac{1}{2} \Delta \theta r^2 \\
	         & = r\Delta\theta r + \frac{1}{2} \Delta \theta \left( \Delta r \right)^2
\end{align*}

Neglecting th \( \left( \Delta r \right)^2 \) term, we get

\[ \Delta A \approx r\Delta\theta\Delta r \]

\begin{prop}
	Suppose \( R \) is described in polar coordinates by \( a \leq r \leq b \) and \( \alpha \leq \theta \leq \beta \), then

	\[ \iint_{R}fdA = \int_{r=a}^{b} \left[ \int_{\theta=\alpha}^{\beta} f\left( r\cos\theta,r\sin\theta \right)r d\theta \right] dr \]
\end{prop}

Here, \( dA = rd\theta dr \) is the infinitesimal area. Also, \( f\left( r\cos\theta,r\sin\theta \right) \) is simply \( f\left( x,y \right) \) in polar coordinates.

\begin{eg}
	Consider a  disc of radius \( t \). Its area is \( \pi t^2 \)

	\begin{figure}[H]
		\centering
		\incfig{disc}
	\end{figure}

	It is described by \( R = \left\{ \left( 1,\theta \right): 0 \leq r \leq t, 0 \leq \theta \leq 2\pi \right\} \). We compute

	\[ \iint_{R}dA = \int_{0}^{t} \left[ \int_{0}^{2\pi} r d\theta \right] dr  \]

	\begin{align*}
		\iint_{R}dA & = \int_{0}^{t} \left[ \int_{0}^{2\pi} r d\theta \right] dr \\
		            & = 2\pi \int_{0}^{t} r dr                                   \\
		            & = 2\pi \left[ \frac{r^2}{2} \right]_{0}^{t}                \\
		            & = \pi t^2
	\end{align*}

\end{eg}

\begin{eg}
	We have \( R \) as before, we want to compute \( \iint_{R}fdA \) where

	\[ f\left( x,y \right)=x^2+y^2 \]

	Using polar coordinates, we get

	\[ f\left( r\cos\theta,r\sin\theta \right) = r^2\left( \cos\theta \right)+r^2+\left( \sin\theta \right)^2 = r^2 \]

	Our double integral is

	\begin{align*}
		\iint_{R}fdA & = \int_{0}^{t} \int_{0}^{2\pi} f\left( r\cos\theta,r\sin\theta\right)r d\theta dr \\
		             & = \int_{0}^{t} \int_{0}^{2\pi} r^3 d\theta dr                                     \\
		             & = 2\pi \int_{0}^{t} r^3 dr                                                        \\
		             & = 2\pi \frac{1}{4} t^{4}                                                          \\
		             & = \frac{\pi}{2}t^{4}
	\end{align*}
\end{eg}

\subsection{Change in variables}

We want to consider general coordinates \( \left( u,v \right) \) defined by

\[ x = x\left( u,v \right), \quad y = y\left( u,v \right) \]

How to integrate with \( \left( u,v \right) \)?

\begin{definition}
	The \textit{jacobian determinant} is defined by

	\[ J\left( u,v \right) = \left| \begin{bmatrix}
			\frac{\partial x}{\partial u} & \frac{\partial x}{\partial v} \\
			\frac{\partial y}{\partial u} & \frac{\partial y}{\partial v}
		\end{bmatrix} \right| = \frac{\partial x}{\partial u}\cdot \frac{\partial y}{\partial v} - \frac{\partial x}{\partial v} \cdot  \frac{\partial y}{\partial u} \]
\end{definition}

\begin{eg}
	Consider the polar coordinates

	\[ x = r\cos\theta, \quad y = r\sin\theta \]

	We want to compute \( J\left( r,\theta \right) \), we have

	\[ J\left( r,\theta \right) = \left| \begin{bmatrix}
			\frac{\partial x}{\partial r} & \frac{\partial x}{\partial \theta} \\
			\frac{\partial y}{\partial r} & \frac{\partial y}{\partial \theta}
		\end{bmatrix} \right| = \left| \begin{bmatrix}
			\cos\theta & -r\sin\theta \\
			\sin\theta & r\cos\theta
		\end{bmatrix} \right| \]

	We obtain

	\begin{align*}
		J\left( r,\theta \right) & = \cos\theta \cdot r\cos\theta - \sin\theta \cdot \left( -r\sin\theta \right) \\
		                         & = r\left( \left( \cos\theta \right)^2+\left( \sin\theta \right)^2 \right)     \\
		                         & = r
	\end{align*}

	This correspons to \( r \) in \( dA = rd\theta dr \).
\end{eg}

The determinant should be intepreted as an area (up to certain signs).

\begin{eg}
	Given \( \vec{v}=\left( a,b \right) \) and \( \vec{u}=\left( 0,d \right) \), consider \( \left( 0,0 \right) \), \( \vec{v}=\left( a,b \right) \), \( \vec{w}=\left( 0,d \right) \), \( \vec{v}+\vec{w} = \left( a,b+d \right) \). These four points define a paralellogram.


	\begin{figure}[H]
		\centering
		\incfig{paralellogram}
	\end{figure}
\end{eg}

we claim that

\[ a = \left| \text{ det} \begin{pmatrix}
		a & b \\
		0 & d
	\end{pmatrix} \right| = \left| ad \right| \]

we consider \( a, b, d > 0 \), then

\[ a = a\left( b+d \right) - \frac{1}{2} ab = \frac{1}{2} ab = ad \]

this is the same as the determinant. more generally, with

\[ \left( 0,0 \right), \vec{v}=\left( a,b \right), \quad \vec{w} = \left( c,d \right), \quad \vec{v}+\vec{w}=\left( a+c, b+d \right) \]

\[ \left| \text{ det} \begin{pmatrix}
		a & b \\
		c & d
	\end{pmatrix} \right| = \left| ab-bc \right| \]

the idea of change of variables can be described by the following figure.


\begin{figure}[h]
	\centering
	\incfig{changevariables}
\end{figure}
\[  \]
\begin{theorem}
	Change of variables.
	\medskip

	Let \( R \) be a region in cartesian coordinates


	\begin{figure}[H]
		\centering
		\incfig{cartesiancoors}
	\end{figure}

	Let \( S \) be the corresponding region in \( \left( u,v \right) \)-coordinates, then

	\[ \iint_{R} fdA = \iint_{S}f\left( x\left( u,v \right),y\left( u,v \right) \right)\left| J\left( u,v \right) \right|dudv \]
\end{theorem}


\end{document}
