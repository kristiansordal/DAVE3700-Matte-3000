\documentclass{article}
% Some basic packages
\usepackage[utf8]{inputenc}
\usepackage[margin=1.2in]{geometry}
\usepackage{textcomp}
\usepackage{url}
\usepackage{graphicx}
\usepackage{float}
\usepackage{enumitem}
\usepackage{standalone}
\usepackage{tcolorbox}
\usepackage{wrapfig}

%color settings
\usepackage{xcolor}
\definecolor{gruvbgdark}{HTML}{1d2021}
\definecolor{gruvtextdark}{HTML}{ebdbb2}
\definecolor{gruvbglight}{HTML}{f9f5d7}
\definecolor{gruvtextlight}{HTML}{3c3836}
% \pagecolor{gruvbgdark}
% \color{gruvtextdark}

% Hide page number when page is empty
\usepackage{emptypage}
\usepackage{subcaption}
\usepackage{multicol}
\usepackage{xcolor}

% Math stuff
\usepackage{amsmath, amsfonts, mathtools, amsthm, amssymb}
% Fancy script capitals
\usepackage{mathrsfs}
\usepackage{cancel}

% Bold math
\usepackage{bm}

% Some shortcuts
\newcommand\N{\ensuremath{\mathbb{N}}}
\newcommand\R{\ensuremath{\mathbb{R}}}
\newcommand\Z{\ensuremath{\mathbb{Z}}}
\renewcommand\O{\ensuremath{\emptyset}}
\newcommand\Q{\ensuremath{\mathbb{Q}}}
\newcommand\C{\ensuremath{\mathbb{C}}}

%Make implies and impliedby shorter
\let\implies\Rightarrow
\let\impliedby\Leftarrow
\let\iff\Leftrightarrow
\let\epsilon\varepsilon

% Add \contra symbol to denote contradiction
% \usepackage{stmaryrd} % for \lightning
% \newcommand\contra{\scalebox{1.5}{$\lightning$}}

% \let\phi\varphi

% Command for short corrections
% Usage: 1+1=\correct{3}{2}

\definecolor{correct}{HTML}{009900}
\newcommand\correct[2]{\ensuremath{\:}{\color{red}{#1}}\ensuremath{\to }{\color{correct}{#2}}\ensuremath{\:}}
\newcommand\green[1]{{\color{correct}{#1}}}

% horizontal rule
% \newcommand\hr{
%     \noindent\rule[0.5ex]{\linewidth}{0.5pt}
% }

% hide parts
\newcommand\hide[1]{}

% Environments
\makeatother
% For box around Definition, Theorem, \ldots
\usepackage{mdframed}
\mdfsetup{skipabove=1em,skipbelow=1em}
\theoremstyle{definition}

\newmdtheoremenv[nobreak=true]{definition}{Definition}
\newtheorem*{eg}{Example}
\newtheorem*{notation}{Notation}
\newtheorem*{previouslyseen}{As previously seen}
\newtheorem*{remark}{Remark}
\newtheorem*{note}{Note}
\newtheorem*{problem}{Problem}
\newtheorem*{observe}{Observe}
\newtheorem*{property}{Property}
\newtheorem*{intuition}{Intuition}
\newmdtheoremenv[nobreak=true]{prop}{Proposition}
\newmdtheoremenv[nobreak=true]{theorem}{Theorem}
\newmdtheoremenv[nobreak=true]{corollary}{Corollary}

% \newtcbtheorem{uctheorem}{Theorem}{uncheckedstyle}{theo}
% End example and intermezzo environments with a small diamond (just like proof
% environments end with a small square)
\usepackage{etoolbox}
\AtEndEnvironment{vb}{\null\hfill$\diamond$}%
\AtEndEnvironment{intermezzo}{\null\hfill$\diamond$}%
% \AtEndEnvironment{opmerking}{\null\hfill$\diamond$}%

% Fix some spacing
% http://tex.stackexchange.com/questions/22119/how-can-i-change-the-spacing-before-theorems-with-amsthm
\makeatletter
\def\thm@space@setup{%
	\thm@preskip=\parskip \thm@postskip=0pt
}


% Exercise 
% Usage:
% \oefening{5}
% \suboefening{1}
% \suboefening{2}
% \suboefening{3}
% gives
% Oefening 5
%   Oefening 5.1
%   Oefening 5.2
%   Oefening 5.3
\newcommand{\oefening}[1]{%
	\def\@oefening{#1}%
	\subsection*{Oefening #1}
}

\newcommand{\suboefening}[1]{%
	\subsubsection*{Oefening \@oefening.#1}
}


% \lecture starts a new lecture (les in dutch)
%
% Usage:
% \lecture{1}{di 12 feb 2019 16:00}{Inleiding}
%
% This adds a section heading with the number / title of the lecture and a
% margin paragraph with the date.

% I use \dateparts here to hide the year (2019). This way, I can easily parse
% the date of each lecture unambiguously while still having a human-friendly
% short format printed to the pdf.

% \usepackage{xifthen}
% \def\testdateparts#1{\dateparts#1\relax}
% \def\dateparts#1 #2 #3 #4 #5\relax{
% 	\marginpar{\small\textsf{\mbox{#1 #2 #3 #5}}}
% }

% \def\@lecture{}%
% \newcommand{\lecture}[3]{
% 	\ifthenelse{\isempty{#3}}{%
% 		\def\@lecture{Lecture #1}%
% 	}{%
% 		\def\@lecture{Lecture #1: #3}%
% 	}%
% 	\subsection*{\@lecture}
% 	% \marginpar{\small\textsf{\mbox{#2}}}
% }



% These are the fancy headers
\usepackage{fancyhdr}
\pagestyle{fancy}

% LE: left even
% RO: right odd
% CE, CO: center even, center odd
% My name for when I print my lecture notes to use for an open book exam.
\fancyhead[LE,RO]{Kristian Sørdal}

\fancyhead[RO,LE]{DAVE3700 - Matte 3000} % Right odd,  Left even
\fancyhead[RE,LO]{\leftmark}          % Right even, Left odd

\fancyfoot[RO,LE]{\thepage}  % Right odd,  Left even
\fancyfoot[RE,LO]{}          % Right even, Left odd
\fancyfoot[C]{\leftmark}     % Center

\makeatother

% Todonotes and inline notes in fancy boxes
\usepackage{todonotes}
\usepackage{tcolorbox}

% Make boxes breakable
\tcbuselibrary{breakable}

% Figure support as explained in my blog post.
\usepackage{import}
\usepackage{xifthen}
\usepackage{pdfpages}
\usepackage{transparent}
\newcommand{\incfig}[1]{%
	\def\svgwidth{\columnwidth}
	\import{./figures/}{#1.pdf_tex}
}

% Fix some stuff
% %http://tex.stackexchange.com/questions/76273/multiple-pdfs-with-page-group-included-in-a-single-page-warning
\pdfsuppresswarningpagegroup=1

\author{Kristian Sørdal}

\begin{document}

\section{Domains, Graphs and Level Sets}

\subsection{Domain of definition}
A function may not be defined for all real numbers.

\begin{eg}
    \( f(x) = \frac{1}{x}  \) is not defined for \( x=0 \)
\end{eg}

\begin{definition}
    The domain of a function \( f \) is the set of numbers for which it is defined. We write the domain of \( f \) as \( D_{f} \).
\end{definition}

For instance, for \( f(x) = \frac{1}{x}\) we have that

\[ D_{f} = \left\{ x \in \R: x \neq\right\} \]

This is the lasrgerst possible domain, we can also consider smaller domains. We have the interval from 1 to 2.

\[ \left[ 1,2 \right] = \left\{ x \in \R: 1 \leq x \leq 2 \right\} \]

\begin{eg}
    Find the largest domain of \( f(x,y) = \frac{1}{y-x}  \). The denominator should be non-zero, we get

    \[ D_{f} = \left\{ \left( x,y \right)\in \R^2:y-x \neq 0 \right\} \]
\end{eg}

\begin{eg}
    Same excersise with \( f(x,y) = \sqrt{y-x^2} \).
    
    Argument: \( y-x^2 \geq 0 \) (because square root). We will then have \( y \geq x^2 \)
\end{eg}

\subsection{Graphs of functions}
The plot of a function \( f \) descrives its behaviour visually. Mathematically, a plot corresponds to the notion of a graph.

\begin{definition}
    The graph of a function \( f(x,y) \) with domain \( D_{f} \) is the set of points \( \left( x,y,z \right) \) such that:

    \[ \left( x,y \right) \in D_{f} \text{ and } z = f(x,y) \]

    We write \( G_{f} \) for the set

    \[ G_{f} = \left\{ \left( x,y,z \right) \in \R^3: \left( x,y \right) \in D_{f} \text{ and } z = f\left( x,y \right) \right\} \]

    The graph of a function of two variables will, in general, be a surface.
\end{definition}        

\begin{eg}
    Lets consider \( f(x,y)=1 \), with domain \( \R^2 \). The graph of \( f \) is:

    \[ G_{f} = \left\{ \left( x,y,1 \right):\left( x,y \right)\R^2 \right\} \] 
    All points have \( z = 1 \), this is a plane. More generally, the graph of\( f(x,y) = ax + by + c \) is a plane with linear dependence on \( x \text{ and } y \).
\end{eg}
\bigskip

\begin{eg}
    Consider the graph of:

    \[ f(x,y) = x^2+y^2,\quad D_{F}= \R^2 \]

    This surface is called a parabloids.
\end{eg}
\bigskip

\begin{eg}
    A sphere of radius \( r \) is defined by

    \[ x^2 + y^2 + z^2 = r^2 \]

    All points \( x,y,z \) satisfy the equation.
\end{eg}

Is this the graph of a function? No!

There is no unique value of \( z \), associated with \( \left( x,y \right) \) because:

\[ z = \pm \sqrt{r^2 - x^2 -y^2} \]


Both satisfy the sphere equation. Lets consider the graph of

\[ f(x,y) = \sqrt{r^2 - x^2 -y^2} \]

With the domain \( x^2+y^2 \leq r^2 \). The grap is a half sphere.

\subsection{Level Sets}

Another way to visualize functions.

\begin{definition}
    A level set of a function \( f(x,y) \) is constant. Essentialy, this is a topographic map.
\end{definition}

\begin{eg}
    Consider the function

    \[ f(x,y) = x^2 + y^2 \]

    The level sets for \( c > 0 \) are circles.

    \[ f(x,y) = x^2 +y^2 = c = \left( \sqrt{c} \right)^2\]

    This is a circle with radius \( \sqrt{c} \).

    Now consider the case \( c < 0 \), then:

    \[ f\left( x,y \right) = x^2 + y^2 = c \]

    which doesn't work, because the level sets are empty.

    For \( c = 0 \), we only have the point \( \left( x,y \right) = \left( 0,0 \right) \). Generally, level sets of \( f(x,y) \) is a curve.

\end{eg}

\end{document}

