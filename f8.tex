\documentclass{article}
% Some basic packages
\usepackage[utf8]{inputenc}
\usepackage[margin=1.2in]{geometry}
\usepackage{textcomp}
\usepackage{url}
\usepackage{graphicx}
\usepackage{float}
\usepackage{enumitem}
\usepackage{standalone}
\usepackage{tcolorbox}
\usepackage{wrapfig}

%color settings
\usepackage{xcolor}
\definecolor{gruvbgdark}{HTML}{1d2021}
\definecolor{gruvtextdark}{HTML}{ebdbb2}
\definecolor{gruvbglight}{HTML}{f9f5d7}
\definecolor{gruvtextlight}{HTML}{3c3836}
% \pagecolor{gruvbgdark}
% \color{gruvtextdark}

% Hide page number when page is empty
\usepackage{emptypage}
\usepackage{subcaption}
\usepackage{multicol}
\usepackage{xcolor}

% Math stuff
\usepackage{amsmath, amsfonts, mathtools, amsthm, amssymb}
% Fancy script capitals
\usepackage{mathrsfs}
\usepackage{cancel}

% Bold math
\usepackage{bm}

% Some shortcuts
\newcommand\N{\ensuremath{\mathbb{N}}}
\newcommand\R{\ensuremath{\mathbb{R}}}
\newcommand\Z{\ensuremath{\mathbb{Z}}}
\renewcommand\O{\ensuremath{\emptyset}}
\newcommand\Q{\ensuremath{\mathbb{Q}}}
\newcommand\C{\ensuremath{\mathbb{C}}}

%Make implies and impliedby shorter
\let\implies\Rightarrow
\let\impliedby\Leftarrow
\let\iff\Leftrightarrow
\let\epsilon\varepsilon

% Add \contra symbol to denote contradiction
% \usepackage{stmaryrd} % for \lightning
% \newcommand\contra{\scalebox{1.5}{$\lightning$}}

% \let\phi\varphi

% Command for short corrections
% Usage: 1+1=\correct{3}{2}

\definecolor{correct}{HTML}{009900}
\newcommand\correct[2]{\ensuremath{\:}{\color{red}{#1}}\ensuremath{\to }{\color{correct}{#2}}\ensuremath{\:}}
\newcommand\green[1]{{\color{correct}{#1}}}

% horizontal rule
% \newcommand\hr{
%     \noindent\rule[0.5ex]{\linewidth}{0.5pt}
% }

% hide parts
\newcommand\hide[1]{}

% Environments
\makeatother
% For box around Definition, Theorem, \ldots
\usepackage{mdframed}
\mdfsetup{skipabove=1em,skipbelow=1em}
\theoremstyle{definition}

\newmdtheoremenv[nobreak=true]{definition}{Definition}
\newtheorem*{eg}{Example}
\newtheorem*{notation}{Notation}
\newtheorem*{previouslyseen}{As previously seen}
\newtheorem*{remark}{Remark}
\newtheorem*{note}{Note}
\newtheorem*{problem}{Problem}
\newtheorem*{observe}{Observe}
\newtheorem*{property}{Property}
\newtheorem*{intuition}{Intuition}
\newmdtheoremenv[nobreak=true]{prop}{Proposition}
\newmdtheoremenv[nobreak=true]{theorem}{Theorem}
\newmdtheoremenv[nobreak=true]{corollary}{Corollary}

% \newtcbtheorem{uctheorem}{Theorem}{uncheckedstyle}{theo}
% End example and intermezzo environments with a small diamond (just like proof
% environments end with a small square)
\usepackage{etoolbox}
\AtEndEnvironment{vb}{\null\hfill$\diamond$}%
\AtEndEnvironment{intermezzo}{\null\hfill$\diamond$}%
% \AtEndEnvironment{opmerking}{\null\hfill$\diamond$}%

% Fix some spacing
% http://tex.stackexchange.com/questions/22119/how-can-i-change-the-spacing-before-theorems-with-amsthm
\makeatletter
\def\thm@space@setup{%
	\thm@preskip=\parskip \thm@postskip=0pt
}


% Exercise 
% Usage:
% \oefening{5}
% \suboefening{1}
% \suboefening{2}
% \suboefening{3}
% gives
% Oefening 5
%   Oefening 5.1
%   Oefening 5.2
%   Oefening 5.3
\newcommand{\oefening}[1]{%
	\def\@oefening{#1}%
	\subsection*{Oefening #1}
}

\newcommand{\suboefening}[1]{%
	\subsubsection*{Oefening \@oefening.#1}
}


% \lecture starts a new lecture (les in dutch)
%
% Usage:
% \lecture{1}{di 12 feb 2019 16:00}{Inleiding}
%
% This adds a section heading with the number / title of the lecture and a
% margin paragraph with the date.

% I use \dateparts here to hide the year (2019). This way, I can easily parse
% the date of each lecture unambiguously while still having a human-friendly
% short format printed to the pdf.

% \usepackage{xifthen}
% \def\testdateparts#1{\dateparts#1\relax}
% \def\dateparts#1 #2 #3 #4 #5\relax{
% 	\marginpar{\small\textsf{\mbox{#1 #2 #3 #5}}}
% }

% \def\@lecture{}%
% \newcommand{\lecture}[3]{
% 	\ifthenelse{\isempty{#3}}{%
% 		\def\@lecture{Lecture #1}%
% 	}{%
% 		\def\@lecture{Lecture #1: #3}%
% 	}%
% 	\subsection*{\@lecture}
% 	% \marginpar{\small\textsf{\mbox{#2}}}
% }



% These are the fancy headers
\usepackage{fancyhdr}
\pagestyle{fancy}

% LE: left even
% RO: right odd
% CE, CO: center even, center odd
% My name for when I print my lecture notes to use for an open book exam.
\fancyhead[LE,RO]{Kristian Sørdal}

\fancyhead[RO,LE]{DAVE3700 - Matte 3000} % Right odd,  Left even
\fancyhead[RE,LO]{\leftmark}          % Right even, Left odd

\fancyfoot[RO,LE]{\thepage}  % Right odd,  Left even
\fancyfoot[RE,LO]{}          % Right even, Left odd
\fancyfoot[C]{\leftmark}     % Center

\makeatother

% Todonotes and inline notes in fancy boxes
\usepackage{todonotes}
\usepackage{tcolorbox}

% Make boxes breakable
\tcbuselibrary{breakable}

% Figure support as explained in my blog post.
\usepackage{import}
\usepackage{xifthen}
\usepackage{pdfpages}
\usepackage{transparent}
\newcommand{\incfig}[1]{%
	\def\svgwidth{\columnwidth}
	\import{./figures/}{#1.pdf_tex}
}

% Fix some stuff
% %http://tex.stackexchange.com/questions/76273/multiple-pdfs-with-page-group-included-in-a-single-page-warning
\pdfsuppresswarningpagegroup=1

\author{Kristian Sørdal}


\begin{document}
\section{Lecture 8}
\subsection{Determining Motion}

Given acceleration \( \vec{a}\left( t \right) \), can we find \( \vec{v}\left( t \right) \text{ and } \vec{s}\left( t \right) \)? Yes, with some initial conditions given, we can. This is done by integration, consider

\[ \vec{v}\left( t \right) = \frac{d \vec{s}}{dt} \]

This is a differential equation for \( \vec{s}\left( t \right) \). To solve it, we integrate both sides in \( t \), from \( t_{1}  \), to \( t_{2}  \). We get

\[ \int_{t_{0} }^{t_{1} } \vec{v}\left( t \right) dt = \int_{t_{0} }^{t_{1} } \frac{d \vec{s}}{dt} dt \]

The fundamental theorem of calculus gives

\[ \vec{s}\left( t_{1}  \right)-\vec{s}\left( t_{0}  \right) = \int_{t_{0} }^{t_{1} } \vec{v}\left( t \right) dt \]

We can determine \( \vec{s}\left( t \right) \) for any \( t \) if we know \( \vec{v}\left( t \right) \) and the initial condition \( \vec{s}\left( t_{0}  \right) \).

\medskip
\begin{eg}
	Consider an object with acceleration

	\[ \vec{a}\left( t \right)=\left( 1,t \right)=\vec{i}-j\vec{j} \]

	We have the following initial conditions

	\[ \vec{s}\left( 0 \right)=\left( 2,0 \right)=2\vec{i} \quad\wedge\quad \vec{v}\left( 0 \right)=0 \]

	We want to determine \( \vec{s}\left( t \right) \). First, to determine \( \vec{v}\left( t \right) \), we compute

	\begin{align*}
		\int_{0}^{t} \vec{a}\left( t \right) dt & = \vec{i}\int_{0}^{t} 1 dt + \vec{j}\int_{0}^{t} t dt \\
		                                        & = t \vec{i} + \frac{1}{2} t^2 \vec{j}
	\end{align*}

	Here \( t_{0} = 0 \), since \( \vec{v}\left( 0 \right) = 0 \), then

	\[ \vec{v}\left( t \right)=\vec{v}_{0} + \int_{0}^{t} \vec{a}\left( t \right) dt = t \vec{i} + \frac{1}{2} t^2 \vec{j} \]

	To determine \( \vec{s} \), we compute

	\begin{align*}
		\int_{0}^{t} \vec{v}\left( t \right) dt & = \vec{i}\int_{0}^{t} t dt + \vec{j}\int_{0}^{t} \frac{1}{2} t^2 dt \\
		                                        & = \frac{1}{2} t^2 \vec{i} + \frac{1}{6} t^3 \vec{j}
	\end{align*}

	Since \( \vec{s}\left( 0 \right) = \left( 2,0 \right) = 2 \vec{i} \), we get

	\[ \vec{s}\left( t \right)=\vec{s}\left( 0 \right) + \int_{0}^{t} \vec{v}\left( t \right) dt = \left( \frac{t^2}{2} + 2  \right)\vec{i} + \frac{1}{6} t^3 \vec{j} \]
\end{eg}
\medskip

\subsection{Arc Length}
A formula that we can concretely use to compute the length of a curve (using a parametrization). Consider a curve \( c \), with parametrization

\[ \vec{s}\left( t \right)=\left( x\left( t \right),y\left( t \right) \right) ,\quad t_{0} \leq t \leq t_{1} \]

\begin{definition}
	The arc length of \( c \) is given by

	\[ S = \int_{t_{0} }^{t_{1} } \sqrt{\left( \frac{d x}{dt} \right)^2 + \left( \frac{d y}{dt} \right)^2} dt \]
\end{definition}

Idea: summing segments of length

\begin{figure}[H]
	\centering
	\def\svgwidth{0.8\textwidth}
	\import{./figures/}{fig16.pdf_tex}
\end{figure}

To compute \( S \), we need a parametrization of \( c \). Does \( S \) depent on this choice? No!
\medskip

\begin{eg}
	Consider the line segment below

	\begin{figure}[H]
		\centering
		\def\svgwidth{0.8\textwidth}
		\import{./figures/}{fig17.pdf_tex}
	\end{figure}

	From elementary geometry, its length is \( \sqrt{1^2+1^2} = \sqrt{2} \). Consider the parametrization

	\[ \vec{s}\left( t \right)\left( t,t \right), \quad 0 \leq t \leq 1 \]

	We have \( \left( x'\left( t \right), y'\left( t \right) \right) = \left( 1,1 \right) \). Then

	\[ S=\int_{0}^{1} \sqrt{1^2+1^2} dt = \sqrt{2}\int_{0}^{1} 1 dt = \sqrt{2}\]

	Instead we choose

	\[ \vec{s}\left( t \right)=\left( 2t,2t \right),\quad 0 \leq t \leq \frac{1}{2}  \]

	We have \( \left( x'\left( t \right),y'\left( t \right) \right) = \left( 2,2 \right) \). Then

	\[ S = \int_{0}^{\frac{1}{2} } \sqrt{2^2+2^2} dt = \sqrt{8}\int_{0}^{\frac{1}{2} } 1 dt = \sqrt{8}\cdot \frac{1}{2} =\sqrt{2} \]
\end{eg}
\medskip

\textbf{Question:} What is the distance crossed up to time \( t \)?

\begin{definition}
	The arc length parameter is

	\[ S\left( t \right)=\int_{t_{0} }^{t_{1} } \sqrt{\left( \frac{d x}{dt} \right)^2 + \left( \frac{d y}{dt} \right)^2} dt \]
\end{definition}

The difference is that we integrate up to \( t \), not \( t_{1}  \). Special case when \( S\left( t_{1}  \right) = S \). There is an important relation between \( S\left( t \right) \text{ and } \vec{v}\left( t \right) \).

\begin{prop}
	We have

	\[ \left| \vec{v}\left( t \right) \right| = \frac{d S}{dt} \]
\end{prop}

\begin{proof}
	The fundamental theorem of calculus states that if

	\[ F(x)=\int_{a}^{x} f(x) dt \rightarrow F'\left( x \right) = f(x) \]

	Applying this to \( S\left( t \right) \), then

	\[ S'\left( t \right)=\frac{d s}{dt}=\sqrt{\left( \frac{d x}{dt} \right)^2+\left( \frac{d y}{dt} \right)^2} \]

	On the other hand, we have that
	\[ \vec{v}\left( t \right)=\left( x'\left( t \right),y'\left( t \right) \right) \quad\wedge\quad \left| \vec{v} \left( t \right)\right| = \left( t \right)=\sqrt{x'\left( t \right)^2 + y'\left( t \right)^2}  \]

	The two expressions coincide.
\end{proof}

\newpage
\begin{eg}
	Consider a circle of radius R, with

	\[ \vec{S}\left( t \right)=\left( R \cos t, R \sin t \right), \quad 0 \leq t \leq 2\pi \]

	\begin{figure}[H]
		\centering
		\def\svgwidth{1\textwidth}
		\import{./figures/}{fig18.pdf_tex}
	\end{figure}

	We want to compute \( S\left( t \right) \), we have

	\[ \sqrt{x'\left( t \right)^2 + y'\left( t \right)^2} = \sqrt{R^2\left( \sin t \right)^2 + R^2\left( \cos t  \right)^2}= R \]

	We want to check that \( \frac{d S}{dt}=\left| \vec{v}\left( t \right) \right| \). We have

	\[ \vec{v}\left( t \right) = \frac{d \vec{S}}{dt}  = \left( -R \sin t, R \cos t \right)\]

	Its length is equal to \( \left| \vec{v}\left( t \right) \right| = R \). Since \( S\left( t \right) = R\left( t \right) \), we see that

	\[ \frac{d S}{dt} = \left| \vec{v}\left( t \right) \right| \]
\end{eg}
\medskip

\end{document}
