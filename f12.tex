\documentclass{article}
% Some basic packages
\usepackage[utf8]{inputenc}
\usepackage[margin=1.2in]{geometry}
\usepackage{textcomp}
\usepackage{url}
\usepackage{graphicx}
\usepackage{float}
\usepackage{enumitem}
\usepackage{standalone}
\usepackage{tcolorbox}
\usepackage{wrapfig}

%color settings
\usepackage{xcolor}
\definecolor{gruvbgdark}{HTML}{1d2021}
\definecolor{gruvtextdark}{HTML}{ebdbb2}
\definecolor{gruvbglight}{HTML}{f9f5d7}
\definecolor{gruvtextlight}{HTML}{3c3836}
% \pagecolor{gruvbgdark}
% \color{gruvtextdark}

% Hide page number when page is empty
\usepackage{emptypage}
\usepackage{subcaption}
\usepackage{multicol}
\usepackage{xcolor}

% Math stuff
\usepackage{amsmath, amsfonts, mathtools, amsthm, amssymb}
% Fancy script capitals
\usepackage{mathrsfs}
\usepackage{cancel}

% Bold math
\usepackage{bm}

% Some shortcuts
\newcommand\N{\ensuremath{\mathbb{N}}}
\newcommand\R{\ensuremath{\mathbb{R}}}
\newcommand\Z{\ensuremath{\mathbb{Z}}}
\renewcommand\O{\ensuremath{\emptyset}}
\newcommand\Q{\ensuremath{\mathbb{Q}}}
\newcommand\C{\ensuremath{\mathbb{C}}}

%Make implies and impliedby shorter
\let\implies\Rightarrow
\let\impliedby\Leftarrow
\let\iff\Leftrightarrow
\let\epsilon\varepsilon

% Add \contra symbol to denote contradiction
% \usepackage{stmaryrd} % for \lightning
% \newcommand\contra{\scalebox{1.5}{$\lightning$}}

% \let\phi\varphi

% Command for short corrections
% Usage: 1+1=\correct{3}{2}

\definecolor{correct}{HTML}{009900}
\newcommand\correct[2]{\ensuremath{\:}{\color{red}{#1}}\ensuremath{\to }{\color{correct}{#2}}\ensuremath{\:}}
\newcommand\green[1]{{\color{correct}{#1}}}

% horizontal rule
% \newcommand\hr{
%     \noindent\rule[0.5ex]{\linewidth}{0.5pt}
% }

% hide parts
\newcommand\hide[1]{}

% Environments
\makeatother
% For box around Definition, Theorem, \ldots
\usepackage{mdframed}
\mdfsetup{skipabove=1em,skipbelow=1em}
\theoremstyle{definition}

\newmdtheoremenv[nobreak=true]{definition}{Definition}
\newtheorem*{eg}{Example}
\newtheorem*{notation}{Notation}
\newtheorem*{previouslyseen}{As previously seen}
\newtheorem*{remark}{Remark}
\newtheorem*{note}{Note}
\newtheorem*{problem}{Problem}
\newtheorem*{observe}{Observe}
\newtheorem*{property}{Property}
\newtheorem*{intuition}{Intuition}
\newmdtheoremenv[nobreak=true]{prop}{Proposition}
\newmdtheoremenv[nobreak=true]{theorem}{Theorem}
\newmdtheoremenv[nobreak=true]{corollary}{Corollary}

% \newtcbtheorem{uctheorem}{Theorem}{uncheckedstyle}{theo}
% End example and intermezzo environments with a small diamond (just like proof
% environments end with a small square)
\usepackage{etoolbox}
\AtEndEnvironment{vb}{\null\hfill$\diamond$}%
\AtEndEnvironment{intermezzo}{\null\hfill$\diamond$}%
% \AtEndEnvironment{opmerking}{\null\hfill$\diamond$}%

% Fix some spacing
% http://tex.stackexchange.com/questions/22119/how-can-i-change-the-spacing-before-theorems-with-amsthm
\makeatletter
\def\thm@space@setup{%
	\thm@preskip=\parskip \thm@postskip=0pt
}


% Exercise 
% Usage:
% \oefening{5}
% \suboefening{1}
% \suboefening{2}
% \suboefening{3}
% gives
% Oefening 5
%   Oefening 5.1
%   Oefening 5.2
%   Oefening 5.3
\newcommand{\oefening}[1]{%
	\def\@oefening{#1}%
	\subsection*{Oefening #1}
}

\newcommand{\suboefening}[1]{%
	\subsubsection*{Oefening \@oefening.#1}
}


% \lecture starts a new lecture (les in dutch)
%
% Usage:
% \lecture{1}{di 12 feb 2019 16:00}{Inleiding}
%
% This adds a section heading with the number / title of the lecture and a
% margin paragraph with the date.

% I use \dateparts here to hide the year (2019). This way, I can easily parse
% the date of each lecture unambiguously while still having a human-friendly
% short format printed to the pdf.

% \usepackage{xifthen}
% \def\testdateparts#1{\dateparts#1\relax}
% \def\dateparts#1 #2 #3 #4 #5\relax{
% 	\marginpar{\small\textsf{\mbox{#1 #2 #3 #5}}}
% }

% \def\@lecture{}%
% \newcommand{\lecture}[3]{
% 	\ifthenelse{\isempty{#3}}{%
% 		\def\@lecture{Lecture #1}%
% 	}{%
% 		\def\@lecture{Lecture #1: #3}%
% 	}%
% 	\subsection*{\@lecture}
% 	% \marginpar{\small\textsf{\mbox{#2}}}
% }



% These are the fancy headers
\usepackage{fancyhdr}
\pagestyle{fancy}

% LE: left even
% RO: right odd
% CE, CO: center even, center odd
% My name for when I print my lecture notes to use for an open book exam.
\fancyhead[LE,RO]{Kristian Sørdal}

\fancyhead[RO,LE]{DAVE3700 - Matte 3000} % Right odd,  Left even
\fancyhead[RE,LO]{\leftmark}          % Right even, Left odd

\fancyfoot[RO,LE]{\thepage}  % Right odd,  Left even
\fancyfoot[RE,LO]{}          % Right even, Left odd
\fancyfoot[C]{\leftmark}     % Center

\makeatother

% Todonotes and inline notes in fancy boxes
\usepackage{todonotes}
\usepackage{tcolorbox}

% Make boxes breakable
\tcbuselibrary{breakable}

% Figure support as explained in my blog post.
\usepackage{import}
\usepackage{xifthen}
\usepackage{pdfpages}
\usepackage{transparent}
\newcommand{\incfig}[1]{%
	\def\svgwidth{\columnwidth}
	\import{./figures/}{#1.pdf_tex}
}

% Fix some stuff
% %http://tex.stackexchange.com/questions/76273/multiple-pdfs-with-page-group-included-in-a-single-page-warning
\pdfsuppresswarningpagegroup=1

\author{Kristian Sørdal}


\begin{document}
\section{Lecture 12}
\subsection{Identities Between Operations}

We have seen three operations defined by \( \nabla \).

\[ \text{Gradient: } \nabla f, \quad \text{Divergence: } \nabla \cdot  \vec{F}, \quad \text{Curl: } \nabla \times  \vec{F} \]

There are many identities, we'll now look at one.

\begin{prop}
	For any scalar field \( f \), we have

	\[ \nabla \times  \left( \nabla f \right)=0 \]
\end{prop}

\begin{proof}
	We have \( \nabla f = \left( f_{x},f_{y},f_{z} \right) \), then

	\[ \nabla \times \left( \nabla f \right) =
		\left|\begin{bmatrix}
			\vec{i}    & \vec{j}    & \vec{k}    \\
			\partial_x & \partial_y & \partial_z \\
			f_{x}      & f_{y}      & f_{z}
		\end{bmatrix}\right| = \left( f_{zy}-f_{yz}, -f_{zx}+f_{xz},f_{yx} - f_{xy} \right)\]

	But partial derivatives can be exchanged. Then we find that \( \nabla \times \left( \nabla f \right)= 0 \)
\end{proof}

We are going to use this when we discuss conservative fields.

\subsection{Line Integrals}

Consider the curve \( C \) with

\[ \vec{r}\left( t \right)= \left( x\left( t \right),y\left( t \right) \right), t_{0} \leq t \leq t_{1}  \]


\begin{figure}[H]
	\centering
	\incfig[0.8]{lineintegrals}
\end{figure}

\[ \Delta x = x\left( t+\Delta t \right)-x\left( t \right), \quad \Delta y = y\left( t+\Delta t \right)-y\left( t \right) \]

The distance between the two points is

\[ \Delta S = \sqrt{\left( \Delta x^2\right) + \left( \Delta y \right)^2 } \]

When \( \Delta t  \) is small, we can use the following approximation

\[ \Delta x \approx \frac{d x}{dt}\Delta t ,\quad \Delta y \approx \frac{d y}{dt}\Delta t \]

Then, for the distance, we get

\[ S = \sqrt{\left( \frac{d x}{dt}\Delta t^2 \right)+\left( \frac{d y}{dt}\Delta t^2 \right)}=\Delta t \sqrt{\left( \frac{d x}{dt} \right)^2+\left( \frac{d y}{dt} \right)^2} \]

Letting \( \Delta t \rightarrow 0 \) leads to the following

\begin{definition}
	The line integral of \( f\left( x,y \right) \) along a curve \( C \) is defined by

	\[ \int_{C}fds = \int_{t_{0} }^{t_{1} } f\left( x\left( t \right),y\left( t \right) \right)\sqrt{x'\left( t \right)^2+y'\left( t \right)^2} dt \]
\end{definition}

\begin{observe}
	Note that \( f \) is restricted to \( \vec{r}\left( t \right)=\left( x\left( t \right),y\left( t \right) \right) \). When \( f=1 \), we recover the arc length.
\end{observe}

\begin{eg}
	Consider \( C \) defined by

	\[ x\left( t \right)=t, \quad y\left( t \right)=0, \quad 0 \leq t \leq 1 \]

	\begin{figure}[H]
		\centering
		\incfig[0.8]{f12eg1}
	\end{figure}

	First, we compute

	\[ \sqrt{x'\left( t \right)^2 + y'\left( t \right)^2}dt = \sqrt{1^2+0^2}=1 \]

	Now, consider \( f\left( x,y \right)=x^2+y \). We want to compute \( \int_{C}fds \). Restricting \( f \) to \( C \) gives

	\[ f\left( x\left( t \right),y\left( t \right) \right) = x\left( t \right)^2+y\left( t \right)^2 = t^2+0 = t^2 \]

	Then we obtain

	\[ \int_{C}fds=\int_{0}^{1} t^2 \cdot 1 dt=\frac{1}{3}  \]
\end{eg}

\begin{note}
	Line integrals can be used to compute the mass of a 1-dimensional object. The curve \( C \) describes the object, and the function \( \int_{C}fds \) is the mass.

	\begin{figure}[H]
		\centering
		\incfig[0.4]{f12note1}
	\end{figure}
\end{note}

\subsection{Parametrization and Orientation}
The next result is as for the arc length.

\begin{prop}
	The integral \( \int_{C}fds \) does not depent on the parametrization of \( C \).
\end{prop}

We will consider a special case

\begin{eg}
	Consider \( C \) with

	\[ \vec{r}\left( t \right)=\left( x\left( t \right),y\left( t \right) \right), t_{0} \leq t \leq t_{1}  \]

	We have

	\[ \text{Start: } A = \left( x\left( t_{0}  \right), y \left( t_{0}  \right) \right), \quad \text{End: } B = \left( x\left( t_{1}  \right), y\left( t_{1}  \right) \right) \]

	We want to go from \( B \) to \( A \).


	\begin{figure}[H]
		\centering
		\incfig{f12eg2}
	\end{figure}

	We can do this in the following manner

	\[ \vec{r}_{\text{opp}}=\left( x\left( -t \right),y\left( -t \right) \right), \quad -t_{1} \leq t \leq -t_{0}  \]


	Note that
	\begin{align*}
		 & \vec{r}_{\text{opp}}\left( -t_{1}  \right)= \left( x\left( t1 \right),y\left( t_{1}  \right) \right) = B \\
		 & \vec{r}_{\text{opp}}\left( -t_{0}  \right)= \left( x\left( t0 \right),y\left( t_{0}  \right) \right) = A \\
	\end{align*}
\end{eg}

\begin{eg}
	Consider the segment \( C \) from \( \left( 0,0 \right) \) to \( \left( 1,0 \right) \). Take \( f\left( x,y \right)=x \). Show that \( \int_{C}fds=\frac{1}{2}  \) using \( \vec{r}\left( t \right) \) and \( \vec{r}_{\text{opp}}\left( t \right) \).

	If \( C \) parametrized by \( \vec{r}\left( t \right) \), we use \( -C \) when considering \( \vec{r}_{\text{opp}}\left( t \right) \). We have

	\[ \int_{C}fds=\int_{-C}fds \]

	The situation will be different for vector fields.
\end{eg}

\subsection{Case of Vector Fields}

Consider the curve with

\begin{align*}
	 & \vec{r}\left( t \right)=\left( x\left( t \right),y\left( t \right) \right), \quad t_{0} \leq t \leq t_{1}     \\
	 & \vec{r'} \left( t \right)=\left( x'\left( t \right),y'\left( t \right) \right) \quad \text{(Velocity vector)}
\end{align*}

\begin{definition}
	The line integral of \( \vec{F} \) along \( C \) is

	\[ \int_{C}\vec{F} \cdot d\vec{r}= \int_{t_{0} }^{t_{1} } \vec{F}\left( \vec{r}\left( t \right) \right) \cdot \vec{r}' dt \]

	Here \( \vec{F}\left( \vec{r}\left( t \right) \right) = \vec{F}\left( x\left( t \right),y\left( t \right) \right) \)
\end{definition}

In physics, we have that \( \vec{F} \) is the force, and \( \int_{C}\vec{F}d \vec{r} \) is the work done by \( \vec{F} \) along \( C \). The elementary case is given by \( W = FS \), or work = force \(  \cdot  \) displacement. More explicitly, we consider

\[ \vec{F}\left( x,y \right)=\left( P\left( x,y \right),Q\left( x,y \right) \right)  \]

Then we have

\[ \vec{F}\left( \vec{r}\left( t \right) \right) \cdot \vec{r}'\left( t \right)= P\left( x\left( t \right),y\left( t \right) \right) \cdot x'\left( t \right)+Q\left( x\left( t \right),y\left( t \right) \right) \cdot  y'\left( t \right) \]

The line integral is then

\[ \int_{C}\vec{F} \cdot d \vec{r} = \int_{t_{0} }^{t_{1} } P\left( x\left( t \right),y\left( t \right) \right) \cdot  x'\left( t \right) dt + \int_{t_{0} }^{t_{1} } Q\left( x\left( t \right),y\left( t \right) \right) \cdot y'\left( t \right) dt \]

We will write the expression as

\[ \int_{C}\vec{F} \cdot d \vec{r} = \int_{C}P dx + \int_{C} Q dy \]

\begin{eg}
	Consider the curve \( C \) with

	\[ \vec{r}\left( t \right)=\left( t,t^2 \right), \quad 0 \leq t \leq 1 \]

	\begin{figure}[H]
		\centering
		\incfig[1]{f12eg3}
	\end{figure}

	We have \( x\left( t \right)=t \) and \( y\left( t \right)=t^2 \), its derivative is

	\[ \vec{r}\left( t \right)=\left( x'\left( t \right),y'\left( t \right) \right)=\left( 1,2t \right) \]

	Now consider the vector field

	\[ \vec{F}\left( x,y \right)=\left( x+y,x \right) \]

	That is \( P = x + y \) and \( Q =x \), when this is restricted to \( C \), we get

	\[ \vec{F}\left( \vec{r}\left( t \right) \right) = \left( x\left( t \right)+y\left( t \right), y\left( t \right) \right) = \left( t+t^2,t \right) \]

	We want to compute

	\[ \vec{F}\left( \vec{r}\left( t \right) \right) \cdot \vec{r}' = \left( t+t^2,t \right) \cdot  \left( 1,2t \right) = \left( 3t^2 + t \right) \]

	We finally obtain

	\[ \int_{C}\vec{F}\cdot d \vec{r} = \int_{0}^{1} \left( 3t^2+t \right) dt = \frac{3}{2}  \]
\end{eg}



\end{document}
