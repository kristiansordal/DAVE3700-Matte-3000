\documentclass{article}
    % Some basic packages
\usepackage[utf8]{inputenc}
\usepackage[margin=1.2in]{geometry}
\usepackage{textcomp}
\usepackage{url}
\usepackage{graphicx}
\usepackage{float}
\usepackage{enumitem}
\usepackage{standalone}
\usepackage{tcolorbox}
\usepackage{wrapfig}

%color settings
\usepackage{xcolor}
\definecolor{gruvbgdark}{HTML}{1d2021}
\definecolor{gruvtextdark}{HTML}{ebdbb2}
\definecolor{gruvbglight}{HTML}{f9f5d7}
\definecolor{gruvtextlight}{HTML}{3c3836}
% \pagecolor{gruvbgdark}
% \color{gruvtextdark}

% Hide page number when page is empty
\usepackage{emptypage}
\usepackage{subcaption}
\usepackage{multicol}
\usepackage{xcolor}

% Math stuff
\usepackage{amsmath, amsfonts, mathtools, amsthm, amssymb}
% Fancy script capitals
\usepackage{mathrsfs}
\usepackage{cancel}

% Bold math
\usepackage{bm}

% Some shortcuts
\newcommand\N{\ensuremath{\mathbb{N}}}
\newcommand\R{\ensuremath{\mathbb{R}}}
\newcommand\Z{\ensuremath{\mathbb{Z}}}
\renewcommand\O{\ensuremath{\emptyset}}
\newcommand\Q{\ensuremath{\mathbb{Q}}}
\newcommand\C{\ensuremath{\mathbb{C}}}

%Make implies and impliedby shorter
\let\implies\Rightarrow
\let\impliedby\Leftarrow
\let\iff\Leftrightarrow
\let\epsilon\varepsilon

% Add \contra symbol to denote contradiction
% \usepackage{stmaryrd} % for \lightning
% \newcommand\contra{\scalebox{1.5}{$\lightning$}}

% \let\phi\varphi

% Command for short corrections
% Usage: 1+1=\correct{3}{2}

\definecolor{correct}{HTML}{009900}
\newcommand\correct[2]{\ensuremath{\:}{\color{red}{#1}}\ensuremath{\to }{\color{correct}{#2}}\ensuremath{\:}}
\newcommand\green[1]{{\color{correct}{#1}}}

% horizontal rule
% \newcommand\hr{
%     \noindent\rule[0.5ex]{\linewidth}{0.5pt}
% }

% hide parts
\newcommand\hide[1]{}

% Environments
\makeatother
% For box around Definition, Theorem, \ldots
\usepackage{mdframed}
\mdfsetup{skipabove=1em,skipbelow=1em}
\theoremstyle{definition}

\newmdtheoremenv[nobreak=true]{definition}{Definition}
\newtheorem*{eg}{Example}
\newtheorem*{notation}{Notation}
\newtheorem*{previouslyseen}{As previously seen}
\newtheorem*{remark}{Remark}
\newtheorem*{note}{Note}
\newtheorem*{problem}{Problem}
\newtheorem*{observe}{Observe}
\newtheorem*{property}{Property}
\newtheorem*{intuition}{Intuition}
\newmdtheoremenv[nobreak=true]{prop}{Proposition}
\newmdtheoremenv[nobreak=true]{theorem}{Theorem}
\newmdtheoremenv[nobreak=true]{corollary}{Corollary}

% \newtcbtheorem{uctheorem}{Theorem}{uncheckedstyle}{theo}
% End example and intermezzo environments with a small diamond (just like proof
% environments end with a small square)
\usepackage{etoolbox}
\AtEndEnvironment{vb}{\null\hfill$\diamond$}%
\AtEndEnvironment{intermezzo}{\null\hfill$\diamond$}%
% \AtEndEnvironment{opmerking}{\null\hfill$\diamond$}%

% Fix some spacing
% http://tex.stackexchange.com/questions/22119/how-can-i-change-the-spacing-before-theorems-with-amsthm
\makeatletter
\def\thm@space@setup{%
	\thm@preskip=\parskip \thm@postskip=0pt
}


% Exercise 
% Usage:
% \oefening{5}
% \suboefening{1}
% \suboefening{2}
% \suboefening{3}
% gives
% Oefening 5
%   Oefening 5.1
%   Oefening 5.2
%   Oefening 5.3
\newcommand{\oefening}[1]{%
	\def\@oefening{#1}%
	\subsection*{Oefening #1}
}

\newcommand{\suboefening}[1]{%
	\subsubsection*{Oefening \@oefening.#1}
}


% \lecture starts a new lecture (les in dutch)
%
% Usage:
% \lecture{1}{di 12 feb 2019 16:00}{Inleiding}
%
% This adds a section heading with the number / title of the lecture and a
% margin paragraph with the date.

% I use \dateparts here to hide the year (2019). This way, I can easily parse
% the date of each lecture unambiguously while still having a human-friendly
% short format printed to the pdf.

% \usepackage{xifthen}
% \def\testdateparts#1{\dateparts#1\relax}
% \def\dateparts#1 #2 #3 #4 #5\relax{
% 	\marginpar{\small\textsf{\mbox{#1 #2 #3 #5}}}
% }

% \def\@lecture{}%
% \newcommand{\lecture}[3]{
% 	\ifthenelse{\isempty{#3}}{%
% 		\def\@lecture{Lecture #1}%
% 	}{%
% 		\def\@lecture{Lecture #1: #3}%
% 	}%
% 	\subsection*{\@lecture}
% 	% \marginpar{\small\textsf{\mbox{#2}}}
% }



% These are the fancy headers
\usepackage{fancyhdr}
\pagestyle{fancy}

% LE: left even
% RO: right odd
% CE, CO: center even, center odd
% My name for when I print my lecture notes to use for an open book exam.
\fancyhead[LE,RO]{Kristian Sørdal}

\fancyhead[RO,LE]{DAVE3700 - Matte 3000} % Right odd,  Left even
\fancyhead[RE,LO]{\leftmark}          % Right even, Left odd

\fancyfoot[RO,LE]{\thepage}  % Right odd,  Left even
\fancyfoot[RE,LO]{}          % Right even, Left odd
\fancyfoot[C]{\leftmark}     % Center

\makeatother

% Todonotes and inline notes in fancy boxes
\usepackage{todonotes}
\usepackage{tcolorbox}

% Make boxes breakable
\tcbuselibrary{breakable}

% Figure support as explained in my blog post.
\usepackage{import}
\usepackage{xifthen}
\usepackage{pdfpages}
\usepackage{transparent}
\newcommand{\incfig}[1]{%
	\def\svgwidth{\columnwidth}
	\import{./figures/}{#1.pdf_tex}
}

% Fix some stuff
% %http://tex.stackexchange.com/questions/76273/multiple-pdfs-with-page-group-included-in-a-single-page-warning
\pdfsuppresswarningpagegroup=1

\author{Kristian Sørdal}

\begin{document}
\section{Lecutre 5}
\subsection{Hessian Matrix}
\begin{eg}
	Consider again the function \( f\left( x,y \right) = x^2-y^2 \).

	\[ f_{xx} = 2, \quad f_{yy} = -2, \quad f_{yx} = 0 \]

	\[ H = \begin{bmatrix}
			2 & 0  \\
			0 & -2
		\end{bmatrix} = \text{det } H = -4 < 0 \]

	Hence \( \left( x_{0} , y_{0}  \right) \) is a saddle point.
\end{eg}
\bigskip

\begin{eg}
	Consider the function \( f\left( x,y \right) = x^2+y^2 \), we have

	\[ \left( f_{x}, f_{y} \right) = \left( 2x,2y \right) \]

	The only critical point is \( \left( x_{0}, y_{0}  \right) = \left( 0,0 \right) \).

	\[ H = \begin{bmatrix}
			2 & 0 \\
			0 & 2
		\end{bmatrix} = \text{det } H = 4 > 0 \]

	Since \( f_{xx} = 2 > 0 \), we conclude that \( \left( 0,0 \right) \) is a local minima. In this case, it is actually a global minimum, because \( f\left( x,y \right) = x^2 + y^2 \geq 0 \).
\end{eg}
\subsection{Global extremal values}
A function can have many maxima and minimas. Usually, we are intrested in the largest and smallest values.

\begin{definition}
	Let \( f\left( x,y \right) \) be with domain \( D_{f} \). Then we have

	\begin{itemize}
		\item \( \left( x_{0} , y_{0}  \right) \) is a global maxima if \( f\left( x_{0} , y_{0}  \right) \geq f\left( x,y \right) \) for all \( \left( x,y \right) \in D_{f} \).

		\item \( \left( x_{0} , y_{0}  \right) \) is a global minima if \( f\left( x_{0} ,y_{0}  \right) \geq f\left( x,y \right) \) for all \( \left( x,y \right)\in D_{f} \).
	\end{itemize}
\end{definition}

Trivial example: for \( f\left( x,y \right) = 1 \), all points are global maxima and minima.


Note that global maxima and minima need not be critical points.


\begin{eg}
	We have \( f(x) = x \) with \( D_{f} = \left[ -1,1 \right] \).

	\begin{figure}[H]
		\centering
		\def\svgwidth{0.5\textwidth}
		\import{./figures/}{figtest.pdf_tex}
		\caption{\( f(x) \)}
		\label{fig:mk}
	\end{figure}

	\begin{itemize}
		\item Global maxima at \( x=1, f\left( 1 \right) = 1 \).
		\item Global minima at \( x=-1, f\left( -1 \right)=-1 \)
	\end{itemize}

	We have no critical points because \( f'\left( x \right) = 1 \neq 0 \). Also not that maxima and minima depend on the chosen domain.


	If we take \( D_{f} = \left[-2,3  \right] \), then

	\[ \text{Max: } x= 3, \quad \text{Min: }x=-2 \]
\end{eg}

\begin{theorem}
	Let \( f \) be continous with domain \( D_{f} \). Suppose \( D_{f} \) is closed and bounded, then there is at least one global maxima, and one global minima.


	\begin{figure}[H]
		\centering
		\def\svgwidth{0.5\textwidth}
		\import{./figures/}{fig2.pdf_tex}
		\caption{The circle is the boundary.}
		\label{fig:The circle is the boundary.}
	\end{figure}

	Some terminology:

	\[ \text{open } = \left\{ x^2+y^2 < 1 \right\} \]
	\[ \text{closed }= \left\{ x^2+y^2 \leq 1 \right\} \]
\end{theorem}

The method for finding maxima and minima is as follows:
\begin{enumerate}
	\item Find critical points of \( f \) in \( D_{f} \), and characterize them.
	\item Study the points that are on the boundary.
	\item Compare them.
\end{enumerate}

\begin{eg}
	Consider \( f\left( x,y \right)=x^2+y^2 \) with domain

	\[ D = \left\{ \left( x,y \right)\in \R: x^2 + y^2 \leq 1 \right\} \]

	The domain in this case is a disc. The red circle is the boundary.


	We compute \( f_{x}=2x, f_{y}=2y \). The only critical point is the origin at \( \left( x_{0} , y_{0}  \right) = \left( 0,0 \right) \). This is a global minium since \( f\left( 0,0 \right) \text{ and } f\left( x,y \right)\geq 0 \).

	Now, lets consider the boundary

	\[ C = \left\{ x^2+y^2=1 \right\} \]

	For any point \( \left( x_{0} ,y_{0}  \right) \) on the circle \( C \), we have

	\[ f\left( x_{0} , y_{0}  \right) = x_{0} ^2 + y_{0} ^2 = 1 \]

	We claim that this point is a global maximum. For any \( \left( x,y \right) \) in domain \( D_{f} \), we have

	\[ f\left( x,y \right) = x^2+y^2 \leq 1 \]

	The value \( f\left( x,y \right)=1 \) is obtain only at the boundary \( C \). Any point on the circle is a global maxima.
\end{eg}

\subsection{Constrained optimization}
In this section, we will discuss how to find maxima and minima of \( f\left(x,y \right) \) with constraint \( g\left( x,y \right) = 0 \). Think of \( g=0 \) ad a budget, or a geometrical constraint.

\begin{eg}
	We want to minimize \( f\left( x,y \right)=x^2+y^2 \) with the constraint \( g\left( x,y \right) = xy-1 = 0 \).

	\begin{figure}[H]
		\centering
		\def\svgwidth{0.5\textwidth}
		\import{./figures/}{fig3.pdf_tex}
		\caption{\( y=\frac{1}{x}  \)}
		\label{fig:}
	\end{figure}

	We have the following strategy

	\begin{itemize}
		\item Solve \( g=0 \) for one variable. For instance \( y= \frac{1}{x}  \).
		\item Consider: \( h(x)=f\left( x,\frac{1}{x}  \right)=x^2+x^{-2} \)
	\end{itemize}

	Now we can study this function of one variable with no constraints. We can proceed as usual.

	\[ \frac{d h}{dt}=2x-2x^{-3} = 0 \]

	This is equivalent to \( x^{4}=1 \). The real solutions are \( x=\pm 1 \). Since \( y=\frac{1}{x}  \), we get the critical points:

	\[ \left( x,y \right) = \left( 1,1 \right), \quad \left( x,y \right) = \left( -1,-1 \right) \]
\end{eg}
\end{document}

