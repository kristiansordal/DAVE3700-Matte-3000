\documentclass{article}
% Some basic packages
\usepackage[utf8]{inputenc}
\usepackage[margin=1.2in]{geometry}
\usepackage{textcomp}
\usepackage{url}
\usepackage{graphicx}
\usepackage{float}
\usepackage{enumitem}
\usepackage{standalone}
\usepackage{tcolorbox}
\usepackage{wrapfig}

%color settings
\usepackage{xcolor}
\definecolor{gruvbgdark}{HTML}{1d2021}
\definecolor{gruvtextdark}{HTML}{ebdbb2}
\definecolor{gruvbglight}{HTML}{f9f5d7}
\definecolor{gruvtextlight}{HTML}{3c3836}
% \pagecolor{gruvbgdark}
% \color{gruvtextdark}

% Hide page number when page is empty
\usepackage{emptypage}
\usepackage{subcaption}
\usepackage{multicol}
\usepackage{xcolor}

% Math stuff
\usepackage{amsmath, amsfonts, mathtools, amsthm, amssymb}
% Fancy script capitals
\usepackage{mathrsfs}
\usepackage{cancel}

% Bold math
\usepackage{bm}

% Some shortcuts
\newcommand\N{\ensuremath{\mathbb{N}}}
\newcommand\R{\ensuremath{\mathbb{R}}}
\newcommand\Z{\ensuremath{\mathbb{Z}}}
\renewcommand\O{\ensuremath{\emptyset}}
\newcommand\Q{\ensuremath{\mathbb{Q}}}
\newcommand\C{\ensuremath{\mathbb{C}}}

%Make implies and impliedby shorter
\let\implies\Rightarrow
\let\impliedby\Leftarrow
\let\iff\Leftrightarrow
\let\epsilon\varepsilon

% Add \contra symbol to denote contradiction
% \usepackage{stmaryrd} % for \lightning
% \newcommand\contra{\scalebox{1.5}{$\lightning$}}

% \let\phi\varphi

% Command for short corrections
% Usage: 1+1=\correct{3}{2}

\definecolor{correct}{HTML}{009900}
\newcommand\correct[2]{\ensuremath{\:}{\color{red}{#1}}\ensuremath{\to }{\color{correct}{#2}}\ensuremath{\:}}
\newcommand\green[1]{{\color{correct}{#1}}}

% horizontal rule
% \newcommand\hr{
%     \noindent\rule[0.5ex]{\linewidth}{0.5pt}
% }

% hide parts
\newcommand\hide[1]{}

% Environments
\makeatother
% For box around Definition, Theorem, \ldots
\usepackage{mdframed}
\mdfsetup{skipabove=1em,skipbelow=1em}
\theoremstyle{definition}

\newmdtheoremenv[nobreak=true]{definition}{Definition}
\newtheorem*{eg}{Example}
\newtheorem*{notation}{Notation}
\newtheorem*{previouslyseen}{As previously seen}
\newtheorem*{remark}{Remark}
\newtheorem*{note}{Note}
\newtheorem*{problem}{Problem}
\newtheorem*{observe}{Observe}
\newtheorem*{property}{Property}
\newtheorem*{intuition}{Intuition}
\newmdtheoremenv[nobreak=true]{prop}{Proposition}
\newmdtheoremenv[nobreak=true]{theorem}{Theorem}
\newmdtheoremenv[nobreak=true]{corollary}{Corollary}

% \newtcbtheorem{uctheorem}{Theorem}{uncheckedstyle}{theo}
% End example and intermezzo environments with a small diamond (just like proof
% environments end with a small square)
\usepackage{etoolbox}
\AtEndEnvironment{vb}{\null\hfill$\diamond$}%
\AtEndEnvironment{intermezzo}{\null\hfill$\diamond$}%
% \AtEndEnvironment{opmerking}{\null\hfill$\diamond$}%

% Fix some spacing
% http://tex.stackexchange.com/questions/22119/how-can-i-change-the-spacing-before-theorems-with-amsthm
\makeatletter
\def\thm@space@setup{%
	\thm@preskip=\parskip \thm@postskip=0pt
}


% Exercise 
% Usage:
% \oefening{5}
% \suboefening{1}
% \suboefening{2}
% \suboefening{3}
% gives
% Oefening 5
%   Oefening 5.1
%   Oefening 5.2
%   Oefening 5.3
\newcommand{\oefening}[1]{%
	\def\@oefening{#1}%
	\subsection*{Oefening #1}
}

\newcommand{\suboefening}[1]{%
	\subsubsection*{Oefening \@oefening.#1}
}


% \lecture starts a new lecture (les in dutch)
%
% Usage:
% \lecture{1}{di 12 feb 2019 16:00}{Inleiding}
%
% This adds a section heading with the number / title of the lecture and a
% margin paragraph with the date.

% I use \dateparts here to hide the year (2019). This way, I can easily parse
% the date of each lecture unambiguously while still having a human-friendly
% short format printed to the pdf.

% \usepackage{xifthen}
% \def\testdateparts#1{\dateparts#1\relax}
% \def\dateparts#1 #2 #3 #4 #5\relax{
% 	\marginpar{\small\textsf{\mbox{#1 #2 #3 #5}}}
% }

% \def\@lecture{}%
% \newcommand{\lecture}[3]{
% 	\ifthenelse{\isempty{#3}}{%
% 		\def\@lecture{Lecture #1}%
% 	}{%
% 		\def\@lecture{Lecture #1: #3}%
% 	}%
% 	\subsection*{\@lecture}
% 	% \marginpar{\small\textsf{\mbox{#2}}}
% }



% These are the fancy headers
\usepackage{fancyhdr}
\pagestyle{fancy}

% LE: left even
% RO: right odd
% CE, CO: center even, center odd
% My name for when I print my lecture notes to use for an open book exam.
\fancyhead[LE,RO]{Kristian Sørdal}

\fancyhead[RO,LE]{DAVE3700 - Matte 3000} % Right odd,  Left even
\fancyhead[RE,LO]{\leftmark}          % Right even, Left odd

\fancyfoot[RO,LE]{\thepage}  % Right odd,  Left even
\fancyfoot[RE,LO]{}          % Right even, Left odd
\fancyfoot[C]{\leftmark}     % Center

\makeatother

% Todonotes and inline notes in fancy boxes
\usepackage{todonotes}
\usepackage{tcolorbox}

% Make boxes breakable
\tcbuselibrary{breakable}

% Figure support as explained in my blog post.
\usepackage{import}
\usepackage{xifthen}
\usepackage{pdfpages}
\usepackage{transparent}
\newcommand{\incfig}[1]{%
	\def\svgwidth{\columnwidth}
	\import{./figures/}{#1.pdf_tex}
}

% Fix some stuff
% %http://tex.stackexchange.com/questions/76273/multiple-pdfs-with-page-group-included-in-a-single-page-warning
\pdfsuppresswarningpagegroup=1

\author{Kristian Sørdal}

\begin{document}
\section{Lecture 4}

\subsection{Directional Derivatives}
We have seen the following

\begin{itemize}
	\item \( f_{x} =  \) the rate of change along the \( x- \)direction.
	\item \( f_{y} =  \) the rate of change along the \( y- \)direction.
\end{itemize}

What about general directions?

\begin{definition}
	Let \( \vec{u} = \left( a,b \right) \), the directional derivative along \( \vec{u} \text{ at } \left( x,y \right) \) is

	\[ D_{\vec{u}}f\left( x,y \right) = \lim_{h\rightarrow 0} \frac{f\left( x+ha,y+hb \right) - f\left( x,y \right)}{h}  \]
\end{definition}

Note that

\[ \vec{u} = \left( 1,0 \right) \rightarrow D_{\vec{u}} = f_{x} \]
\[ \vec{u} = \left( 1,0 \right) \rightarrow D_{\vec{u}} = f_{y} \]

To compare directions, we take \( \left| \vec{t} \right| = 1 \). Here \( \vec{u} \) is the length of \( \vec{u} \), that is

\[ \left| \vec{u} \right|=\sqrt{\vec{u}\cdot \vec{u}} \]

\begin{prop}
	We have the following result

	\[ D_{\vec{u}}\cdot  f = \nabla f \cdot  \vec{u} \]
\end{prop}


\begin{proof}
	Consider the following function

	\[ g\left( t \right) = f\left( x+ta, y+tb \right) \]

	Its derivative at \( t=0 \) is

	\begin{align*}
		\frac{d g}{dt}\left( 0 \right) & =\lim_{h\rightarrow 0} \frac{g\left( h \right)-g\left( 0 \right)}{h}            \\
		                               & = \lim_{h\rightarrow 0} \frac{f\left( x+ha,y+hb \right)-f\left( x,y \right)}{h} \\
		                               & = D_{\vec{u}} \cdot f\left( x,y \right)
	\end{align*}

	On the other hand, using the chain rule, we get

	\begin{align*}
		\frac{d g}{dt}\left( 0 \right) & = \frac{\partial f}{\partial x}\frac{d \left( x+ta \right)}{dt} + \frac{\partial f}{\partial y}\frac{d \left( y+tb \right)}{dt} \\
		                               & = \frac{\partial f}{\partial x}a+\frac{\partial f}{\partial y}b = \nabla f \cdot  \vec{u}
	\end{align*}
\end{proof}

We can now state another property of \( \nabla f \). The direction where \( f \) changes the most.

\begin{prop}
	\( \left| D_{\vec{u}}\cdot f \right| \) is the largest when \( \vec{u} \) is paralell to \( \nabla f \).
\end{prop}
\begin{proof}
	Recall that, given two vectors, \( \vec{v} \) and \( \vec{w} \), we have that

	\[ \vec{v}\cdot \vec{w} = \left| \vec{v} \right|\cdot \left| \vec{w} \right|\cdot \cos \alpha \]


	When is \( \left| \vec{v}\cdot \vec{w} \right| \) the largest? We have

	\[ \left| \vec{v}\cdot \vec{w} \right| = \left| \vec{v} \right| \cdot  \left| \vec{w} \right|\cdot \left| \cos\alpha \right|, \quad \left| \cos\alpha \right|\leq1\]

	It is the largest when the following condition is true

	\[ \left| \cos\alpha \right| = 1, \quad \alpha=0 \vee \pi \]

	Which means that the vectores are pointing in the same, or opposite direction. Applying this to \( D_{\vec{u}}\cdot  f \), we get


	\begin{align*}
		\left| D_{\vec{u}} \cdot  f\right| & = \left| \nabla f \cdot  \vec{u} \right|                                              \\
		                                   & = \left| \nabla f \right| \cdot \left| \vec{u} \right|\cdot \left| \cos\alpha \right|
	\end{align*}

	For fixed values of \( \left| \vec{u} \right|  \), this is the largest when \(\alpha = 0 \vee \pi \). That is \( \nabla f \) and \( \vec{u} \) are paralell.
\end{proof}

\subsection{Critical Points}
How do we find the maxima and minima of \( f\left( x \right) \)? Lets take a look at \( f'\left( x_{0}  \right) = 0 \).

\begin{definition}
	We say that \( x_{0} ,y_{0}  \) is a critical point of \( f \) if:

	\[ \frac{\partial f}{\partial x}\left( x_{0} ,y_{0}  \right) = 0 \quad\wedge\quad \frac{\partial f}{\partial y}\left( x_{0} , y_{0}  \right)= 0 \]
\end{definition}

Similarily for

\end{document}
