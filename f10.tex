\documentclass{article}
% Some basic packages
\usepackage[utf8]{inputenc}
\usepackage[margin=1.2in]{geometry}
\usepackage{textcomp}
\usepackage{url}
\usepackage{graphicx}
\usepackage{float}
\usepackage{enumitem}
\usepackage{standalone}
\usepackage{tcolorbox}
\usepackage{wrapfig}

%color settings
\usepackage{xcolor}
\definecolor{gruvbgdark}{HTML}{1d2021}
\definecolor{gruvtextdark}{HTML}{ebdbb2}
\definecolor{gruvbglight}{HTML}{f9f5d7}
\definecolor{gruvtextlight}{HTML}{3c3836}
% \pagecolor{gruvbgdark}
% \color{gruvtextdark}

% Hide page number when page is empty
\usepackage{emptypage}
\usepackage{subcaption}
\usepackage{multicol}
\usepackage{xcolor}

% Math stuff
\usepackage{amsmath, amsfonts, mathtools, amsthm, amssymb}
% Fancy script capitals
\usepackage{mathrsfs}
\usepackage{cancel}

% Bold math
\usepackage{bm}

% Some shortcuts
\newcommand\N{\ensuremath{\mathbb{N}}}
\newcommand\R{\ensuremath{\mathbb{R}}}
\newcommand\Z{\ensuremath{\mathbb{Z}}}
\renewcommand\O{\ensuremath{\emptyset}}
\newcommand\Q{\ensuremath{\mathbb{Q}}}
\newcommand\C{\ensuremath{\mathbb{C}}}

%Make implies and impliedby shorter
\let\implies\Rightarrow
\let\impliedby\Leftarrow
\let\iff\Leftrightarrow
\let\epsilon\varepsilon

% Add \contra symbol to denote contradiction
% \usepackage{stmaryrd} % for \lightning
% \newcommand\contra{\scalebox{1.5}{$\lightning$}}

% \let\phi\varphi

% Command for short corrections
% Usage: 1+1=\correct{3}{2}

\definecolor{correct}{HTML}{009900}
\newcommand\correct[2]{\ensuremath{\:}{\color{red}{#1}}\ensuremath{\to }{\color{correct}{#2}}\ensuremath{\:}}
\newcommand\green[1]{{\color{correct}{#1}}}

% horizontal rule
% \newcommand\hr{
%     \noindent\rule[0.5ex]{\linewidth}{0.5pt}
% }

% hide parts
\newcommand\hide[1]{}

% Environments
\makeatother
% For box around Definition, Theorem, \ldots
\usepackage{mdframed}
\mdfsetup{skipabove=1em,skipbelow=1em}
\theoremstyle{definition}

\newmdtheoremenv[nobreak=true]{definition}{Definition}
\newtheorem*{eg}{Example}
\newtheorem*{notation}{Notation}
\newtheorem*{previouslyseen}{As previously seen}
\newtheorem*{remark}{Remark}
\newtheorem*{note}{Note}
\newtheorem*{problem}{Problem}
\newtheorem*{observe}{Observe}
\newtheorem*{property}{Property}
\newtheorem*{intuition}{Intuition}
\newmdtheoremenv[nobreak=true]{prop}{Proposition}
\newmdtheoremenv[nobreak=true]{theorem}{Theorem}
\newmdtheoremenv[nobreak=true]{corollary}{Corollary}

% \newtcbtheorem{uctheorem}{Theorem}{uncheckedstyle}{theo}
% End example and intermezzo environments with a small diamond (just like proof
% environments end with a small square)
\usepackage{etoolbox}
\AtEndEnvironment{vb}{\null\hfill$\diamond$}%
\AtEndEnvironment{intermezzo}{\null\hfill$\diamond$}%
% \AtEndEnvironment{opmerking}{\null\hfill$\diamond$}%

% Fix some spacing
% http://tex.stackexchange.com/questions/22119/how-can-i-change-the-spacing-before-theorems-with-amsthm
\makeatletter
\def\thm@space@setup{%
	\thm@preskip=\parskip \thm@postskip=0pt
}


% Exercise 
% Usage:
% \oefening{5}
% \suboefening{1}
% \suboefening{2}
% \suboefening{3}
% gives
% Oefening 5
%   Oefening 5.1
%   Oefening 5.2
%   Oefening 5.3
\newcommand{\oefening}[1]{%
	\def\@oefening{#1}%
	\subsection*{Oefening #1}
}

\newcommand{\suboefening}[1]{%
	\subsubsection*{Oefening \@oefening.#1}
}


% \lecture starts a new lecture (les in dutch)
%
% Usage:
% \lecture{1}{di 12 feb 2019 16:00}{Inleiding}
%
% This adds a section heading with the number / title of the lecture and a
% margin paragraph with the date.

% I use \dateparts here to hide the year (2019). This way, I can easily parse
% the date of each lecture unambiguously while still having a human-friendly
% short format printed to the pdf.

% \usepackage{xifthen}
% \def\testdateparts#1{\dateparts#1\relax}
% \def\dateparts#1 #2 #3 #4 #5\relax{
% 	\marginpar{\small\textsf{\mbox{#1 #2 #3 #5}}}
% }

% \def\@lecture{}%
% \newcommand{\lecture}[3]{
% 	\ifthenelse{\isempty{#3}}{%
% 		\def\@lecture{Lecture #1}%
% 	}{%
% 		\def\@lecture{Lecture #1: #3}%
% 	}%
% 	\subsection*{\@lecture}
% 	% \marginpar{\small\textsf{\mbox{#2}}}
% }



% These are the fancy headers
\usepackage{fancyhdr}
\pagestyle{fancy}

% LE: left even
% RO: right odd
% CE, CO: center even, center odd
% My name for when I print my lecture notes to use for an open book exam.
\fancyhead[LE,RO]{Kristian Sørdal}

\fancyhead[RO,LE]{DAVE3700 - Matte 3000} % Right odd,  Left even
\fancyhead[RE,LO]{\leftmark}          % Right even, Left odd

\fancyfoot[RO,LE]{\thepage}  % Right odd,  Left even
\fancyfoot[RE,LO]{}          % Right even, Left odd
\fancyfoot[C]{\leftmark}     % Center

\makeatother

% Todonotes and inline notes in fancy boxes
\usepackage{todonotes}
\usepackage{tcolorbox}

% Make boxes breakable
\tcbuselibrary{breakable}

% Figure support as explained in my blog post.
\usepackage{import}
\usepackage{xifthen}
\usepackage{pdfpages}
\usepackage{transparent}
\newcommand{\incfig}[1]{%
	\def\svgwidth{\columnwidth}
	\import{./figures/}{#1.pdf_tex}
}

% Fix some stuff
% %http://tex.stackexchange.com/questions/76273/multiple-pdfs-with-page-group-included-in-a-single-page-warning
\pdfsuppresswarningpagegroup=1

\author{Kristian Sørdal}


\begin{document}
\section{Lecture 10}

\subsection{Curves in polar form}
Polar coordinates are an alternative description to cartesian coordinates \( \left( x,y \right) \).

\begin{definition}
	The polar coordinates \( r,\theta \) are defined by

	\[ x = r \cos \theta, \quad y = r \sin \theta \]

	Their range is respectively

	\[ 0 \leq r \leq \infty, \quad 0 \leq \theta \leq 2\pi \]

	Geometrical meaning is that \( r \) is the distance from the origin, and \( \theta \) is the angle.

	\begin{figure}[H]
		\centering
		\def\svgwidth{1\textwidth}
		\import{./figures/}{fig25.pdf_tex}
	\end{figure}
\end{definition}

From \( \left( x,y \right) \) to \( \left( r,\theta \right) \), we can use

\[ r^2=x^2+y^2 \]

We can describe curves using \( \left( r,\theta \right) \). The idea is to give \( r \) as a function of \( \theta \). The curve will be "traced" as we vary \( \theta \). It is an analogue of \( y = f\left( x \right) \).

\begin{figure}[H]
	\centering
	\def\svgwidth{1\textwidth}
	\import{./figures/}{fig26.pdf_tex}
\end{figure}

\begin{eg}
	Consider the curve

	\[ r\left( \theta \right) = 1, \quad 0 \leq \theta \leq 2\pi \]

	What curve is it? All points have distance 1 from origin \( \left( r=1 \right) \)

	Using \( r^2=x^2+y^2 \), we find that \( x^2+y^2=1 \). We have a circle of radius 1.
\end{eg}

\begin{eg}
	The next curve is called the cardioid, it is defined by

	\[ r\left( \theta \right)=1 - \cos \theta, \quad 0 \leq \theta \leq 2\pi \]

	It is described in cartesian coordinates by

	\[ \left( x^2+y^2+x \right)^2=x^2+y^2 \]

	Polar coordinates work best in the prescence of spherical symmetry. The length of \( C \) can be computed using polar coordinates.
\end{eg}

\begin{prop}
	Let \( C \) be given in polar form by

	\[ r = r\left( \theta \right), \quad \alpha \leq \theta \leq \beta \]

	Then its arc length can be computed by

	\[ S = \int_{\alpha}^{\beta} \sqrt{\left( \frac{d r}{d\theta} \right)^2 + r^2} d\theta \]
\end{prop}

\begin{proof}
	Parametrize \( C \) by

	\[ \vec{r}\left( \theta \right)=\left( x\left( \theta \right),y\left( \theta \right) \right), \quad \alpha \leq \theta \leq \beta \]

	where we set

	\[ x\left( \theta \right) = r\left( \theta\right)\cos \theta, \quad y \left( \theta \right)=r \left( \theta \right)\sin \theta \]

	The derivates are, with \( r' = \frac{d r}{d\theta} \)

	\[ \frac{d x}{d\theta}=r'\cos\theta -r\sin\theta, \quad \frac{d y}{d\theta}=r'\sin\theta+r\cos\theta \]

	After some computation we get

	\[ \left( x' \right)^2+\left( y' \right)^2 = r^2+\left( \frac{d r}{d\theta} \right)^2 \]

	Therefore we get

	\begin{align*}
		S & = \int_{\alpha}^{\beta} \sqrt{\left( x' \right)^2 + \left( y' \right)^2} d\theta \\
		  & = \int_{\alpha}^{\beta} \sqrt{r^2+\left( \frac{d r}{d\theta} \right)^2} d\theta
	\end{align*}
\end{proof}

\begin{eg}
	We have a circle given by

	\[ r\left( \theta \right)= R, \quad 0 \leq \theta \leq 2\pi  \]

	We have \( \frac{d r}{d\theta}=0 \), then

	\[ S = \int_{0}^{2\pi} \sqrt{R^2+0^2}d \theta=R \int_{0}^{2\pi} d\theta =  2\pi R\]

	This gives us the circumference of the circle.
\end{eg}

\subsection{Areas In Polar Form}

We want to compute the area inside a closed curve in polar form.

\begin{figure}[H]
	\centering
	\def\svgwidth{0.85\textwidth}
	\import{./figures/}{fig27.pdf_tex}
\end{figure}

Basic observation:

\begin{figure}[H]
	\centering
	\def\svgwidth{1\textwidth}
	\import{./figures/}{fig28.pdf_tex}
\end{figure}

Add small regions with angle \( \Delta\theta \) and area \( \frac{1}{2} r^2\Delta\theta \).

\begin{prop}
	Consider a closed curve described by

	\[ r = r\left( \theta \right), \quad \alpha \leq \theta \leq \beta \]

	Then its area is

	\[ A = \frac{1}{2} \int_{\alpha}^{\beta} r\left( \theta \right)^2 d\theta \]
\end{prop}

\begin{eg}
	We have a circle of radius \( R \)

	\[ r\left( \theta \right)= R, \quad 0 \leq \theta \leq 2\pi  \]

	we get

	\[ A = \frac{1}{2} \int_{0}^{2\pi} R^2 d\theta = \frac{1}{2} R^2 \int_{0}^{2\pi}  d\theta =\pi R^2\]
\end{eg}

\begin{eg}
	Consider the cardioid

	\[ r\left( \theta \right) = 1 - \cos\theta, \quad 0 \leq \theta \leq 2\pi \]

	The area is given by:


	\begin{align*}
		A & = \int_{0}^{2\pi} \left( 1-\cos\theta \right)^2 d\theta                                         \\
		  & = \frac{1}{2} \int_{0}^{2\pi} \left( 1-2\cos\theta +\left( \cos\theta \right)^2 \right) d\theta
	\end{align*}

	To compute this, we use

	\[ \int \cos\theta d\theta = \sin\theta+C, \quad \int\left( \cos\theta \right)^2 d\theta=\frac{1}{2} \theta + \frac{1}{2} \sin \theta\cos\theta+C \]

	Finally, we obtain

	\[ A = \frac{1}{2} \cdot 2\pi + 0 + \frac{1}{2} \cdot  \frac{1}{2} 2\pi = \frac{3}{2} \pi \]
\end{eg}

\section{Scalar and Vector Fields}
\begin{idea}
	A field describes a property of a region \( R \)

	\begin{figure}[H]
		\centering
		\def\svgwidth{1\textwidth}
		\import{./figures/}{fig29.pdf_tex}
	\end{figure}
\end{idea}

Mathematically described by

\[ f:\R^{m}\rightarrow\R^{n}, \quad m \text{ inputs, } n \text{ outputs} \]

For scalar fields the output is a scalar. For vector fields the output is a vector.

\begin{eg}
	The temperature is a scalar field.

	\[ T:\left( x,y,z \right)\rightarrow T\left( x,y,z \right) \]

	The wind velocity is a vector field

	\[ \vec{W}:\left( x,y,z \right)\rightarrow \vec{W}\left( x,y,z \right) \]
\end{eg}

\begin{notation}
	The notation for vector fields:

	\begin{align*}
		\vec{F}\left( x,y,z \right) & =\left( P\left( x,y,z \right),Q\left( x,y,z \right),R\left( x,y,z \right) \right) \\
		                            & =P(x,y,z)\vec{i} + Q(x,y,z)\vec{j} + R(x,y,z)\vec{k}
	\end{align*}

	\( P,Q,R \) are componente functions.
\end{notation}

In 2D we visualize vector fields by vector plots. For instance, take

\[ \vec{F}\left( x,y \right)  = \left( 0,y \right)\]

\begin{figure}[H]
	\centering
	\def\svgwidth{1\textwidth}
	\import{./figures/}{fig30.pdf_tex}
\end{figure}

A vector field \( \vec{F} \) and a scalar field \( f \) can be related as follows

\begin{definition}
	If \( \vec{F} = \Delta f \), we say that \( \vec{F} \) is a gradient field, and \( f \) is a potential.
\end{definition}

\subsection{Gradient, Divergence and Curl}

These are operation defined in terms of the formal vector

\[ \nabla  = \left( \frac{\partial }{\partial x},\frac{\partial }{\partial y},\frac{\partial }{\partial z} \right) = \frac{\partial }{\partial x}\vec{i} + \frac{\partial }{\partial y}\vec{j} + \frac{\partial }{\partial z}\vec{k} \]

\begin{definition}
	The gradient of a scalar field \( f \) is

	\[ \text{grad }f = \nabla f = \left( \frac{\partial f}{\partial x}, \frac{\partial f}{\partial y},\frac{\partial f}{\partial z} \right) \]

	The output is a vector.
	\medskip

	The divergence of a vector field \( \vec{F}\left( P,Q,R \right) \) is

	\[ \text{div } \vec{F} = \nabla \cdot  \vec{F} = \frac{\partial P}{\partial x} + \frac{\partial Q}{\partial y} + \frac{\partial R}{\partial z} \]

	The output is a scalar field.
	\medskip

	The curl of a vector field \( \vec{F} \) is

	\[ \text{curl } \vec{F} = \nabla \times \vec{F} = \left| \begin{bmatrix}
			\vec{i}      & \vec{j}      & \vec{k}      \\
			\partial_{x} & \partial_{y} & \partial_{z} \\
			P            & Q            & R
		\end{bmatrix} \right| \]



\end{definition}

These operations can all be obtained from \( \nabla \).

\begin{table}[H]
	\begin{center}
		\begin{tabular}[c]{llll}
			Operation  & Input  & Output & Symbol                \\
			\hline
			Gradient   & Scalar & Vector & \( \nabla f \)        \\
			Divergence & Vector & Scalar & \( \nabla \cdot f \)  \\
			Curl       & Vector & Vector & \( \nabla \times f \) \\
		\end{tabular}
	\end{center}
\end{table}


\end{document}



